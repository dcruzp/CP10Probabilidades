\documentclass[12pt]{article}

\usepackage{amsmath}

\begin{document}
	\title{Respuesta de la Clase Pr\'actica 10}
	\author{Daniel De La Cruz Prieto}
	\date{\today}
	
    \maketitle
    
    \section*{Ejercicio 1 } 


    \subsection*{ 1-a) La funci\'on de densidad de un vector aleatorio continuo es :}

    \begin{equation*}
        f\left(x,y\right) = \begin{cases}
    
        xe^{-\left(X+Y\right)} & \mbox{$x\leq0$ y $y\leq 0$}
        \\
               0             & \mbox{en otro caso}
    
       \end{cases}
    \end{equation*}


    \begin{flushleft}
        Tenemos $F_{\left(X\right)}$ la podemos dobtener de la siguiente expresi\'on
    \end{flushleft}

   	\begin{equation*} 
	    \fbox{$
	   	F_{X} =  \displaystyle \lim_{y \to + \infty} F\left(x,y\right) = \displaystyle  F_{\left(X,Y\right)} \left(x,+\infty\right) = \int_{0}^{x}\int_{0}^\infty f\left(u,y\right) dy du	   	
	   	$}
    \end{equation*} 

	\begin{equation*} 
	    \begin{array}{rcl}
	        \to\int_{0}^{x} \int_{0}^{+\infty} f \left(u,y\right) dy du & = &  \int_{0}^{x} \left(\int_{o}^{+ \infty}xe^{-\left(x + y\right)} dy\right)du
	        \\
	        \\
	        \to \int_{0}^{+ \infty} xe^{-\left(x+y\right)}dy & = & xe ^{-\left(x\right)} \int_{0}^{+ \infty} e^{-\left(y\right)}dy = \displaystyle xe^{-\left(x\right)} * \lim_{A\to + \infty} \int_{0}^{A} e^{-\left(y\right)} dy
	        \\
	        \\
	        & = &\displaystyle xe^{-\left(x\right)}  \lim_{A\to+\infty} \left[ -e^{\left(-y\right)}\right]\vert_{0}^{A}   = \displaystyle xe^{-\left(x\right)}* \lim_{A\to+\infty} \left[ -e^{-A} + e^0\right]
	        \\
	        \\
            & = & xe^{-\left(x\right)}
            \\
	    \end{array}
    \end{equation*}        
  
    Entonces :

    \begin{equation*}
        \int_{o}^{+\infty} ue^{-u}du = \int_{0}^{x}ue^{-u} = \int_{o}^{x}   ue^{-u} du
    \end{equation*}
    
    Luego teniendo que :
    
    \begin{equation*}
       \fbox{$
        \int xe^{-x}d = -\left(x+1\right)e^{-x}
        $}
    \end{equation*}

    \begin{equation*}
        \begin{array}{rcl}
            \int_{o}^{x} ue^{-u}du & = & \displaystyle  [-\left(u+1\right)e^{-u}]\vert_{o}^{x}
            \\
            \\
            \int_{o}^{x} ue^{-u}du & = & \displaystyle -\left(x+1\right)e^{-x} - [-\left(0+1\right)e^{-0}]
            \\
            \\
            & = & -\left(x+1\right)e^{-x}+1
            \\
            \\
            \int_{o}^{x} ue^{-u}du & = & 1 -\left(x+1\right)e^{-x} 
            \\
        \end{array}
    \end{equation*}

    Por lo tanto :

    \begin{equation*}
       \fbox{$F_{X} = 1 - \left(x+1\right)e^{-x}$}
    \end{equation*}

    Dada la ecuaci\'on siguinte para calcular $F_{Y}$ :

    \begin{equation*} 
        \fbox{$
            F_{Y} = \displaystyle\lim_{x\to+\infty} F_{\left(x,y\right)}\left(+\infty,x\right)=\int_{-\infty}^{y} \int_{-\infty}^{+\infty} f\left(x,u\right) dx du
        $}
    \end{equation*} 

    \begin{equation*}
        \begin{array}{rcl}
            \int_{0}^{y}\int_{0}^{+\infty}xe^{-\left(x+u\right)}dx du & = & 
            \int_{0}^{y} \left(\int_{0}^{+\infty}xe^{-\left(x+u\right)}dx\right)du 
            \\
            \\
            \int_{0}^{y}\int_{0}^{+\infty}xe^{-\left(x+u\right)}dx du & = & \int_{0}^{+\infty}xe^{-x}e^{-u}dx = e^{-u}\int_{0}^{+\infty}xe^{-x}dx 
            \\
            \\
            \to\int_{0}^{+\infty}xe^{-\left(x+u\right)dx du} & = & \int_{0}^{+\infty}xe^{-x}e^{-u}dx = e^{-u}\int_{0}^{+\infty}xe^{-x}dx
            \\
        \end{array}
    \end{equation*}
    Tenamos que :
      \begin{equation*}
         \fbox{$\int xe^{-x}dx = -\left(x+1\right)e^{-x}$}
      \end{equation*}
    Por lo tanto :
    \begin{equation*}
        \begin{array}{rcl}
            \to\int_{0}^{+\infty}xe^{-x} dx & = & \displaystyle \lim_{A\to+\infty}\int_{0}^{A}xe^{-x}dx
            \\
            \\
            \to\int_{0}^{+\infty}xe^{-x} dx & = &  \displaystyle \lim_{A\to + \infty}[-\left(x+1\right)e^{-x}] \vert_{0}^{A} 
            \\
            \\
            \to\int_{0}^{+\infty}xe^{-x} dx & = &  \displaystyle\lim{A\to+\infty} -\left(A+1\right)e^{-A}+1
            \\
            \\
            \to\int_{0}^{+\infty}xe^{-x} dx & = &  \displaystyle \lim_{A\to+\infty}-Ae^{-A}-e^{-A}+1
            \\
            \\
            \to\int_{0}^{+\infty}xe^{-x} dx & = &  \displaystyle\lim_{A\to+\infty}-Ae^{-A}-\lim_{A\to+\infty}e^{-A}+1
            \\
            \\
            \to\int_{0}^{+\infty}xe^{-x} dx & = &  1
        \end{array}
    \end{equation*}

    Luego comprobando que:

    \begin{equation*}
        \begin{array}{rcl}
            \int_{0}^{y}\left(\int_{0}^{+\infty}xe^{-\left(x+u\right)}dx\right)du & = & \int_{0}^{y} e^{-u }du 
            \\
            \\
            \int_{0}^{y}\left(\int_{0}^{+\infty}xe^{-\left(x+u\right)}dx\right)du & = & -e^{-4}\vert_{o}^{y}
            \\
            \\
            \int_{0}^{y}\left(\int_{0}^{+\infty}xe^{-\left(x+u\right)}dx\right)du & = & -e^{-y}+e^{-0}
            \\
            \\
            \int_{0}^{y}\left(\int_{0}^{+\infty}xe^{-\left(x+u\right)}dx\right)du & = & 1 -e^{-y}
        \end{array}
    \end{equation*}

    Finalmente tenemos que :

    \begin{equation*}
        \fbox{$
        F_{Y}= 1 - e^{-y}
        $}
    \end{equation*}


    \subsection*{1-b) Determinar si las variables $X$ y $Y$ son independientes y $\rho \left(X,Y\right)$}
    
    \begin{flushleft}
        Para calcular $\rho \left(X,Y\right)$  tenemos que calcular primero : $E\left[XY\right]$ , $EX$, $EY$ , $EX^2$ y $EY^2$   
    \end{flushleft}
    

    \begin{flushleft}
        Primero vamos a calcular $E\left[XY\right]$ para eso tenemos: 
    \end{flushleft}

    \begin{equation*}
        \fbox{$
        E\left[XY\right] = \displaystyle \int_{- \infty}^{+ \infty} \int_{-\infty}^{+\infty} xy f\left(x,y\right) \,\mathrm{d}y  \,\mathrm{d}x  = \int_{0}^{+\infty} \int_{0}^{+\infty} x^2ye^{-\left(x+y\right)} \,\mathrm{d}y  \,\mathrm{d}x 
        $}
    \end{equation*}

    \begin{flushleft}
        Primero calculo $\int_{0}^{+ \infty} x^2ye^{-\left(x+y\right)} \,\mathrm{d}y $ 
    \end{flushleft}

    \begin{equation*}
        \begin{array}{rcl}
            \displaystyle \int_{0}^{+ \infty} x^2ye^{-\left(x+y\right)} \,\mathrm{d}y  &= & \displaystyle  x^2 e^{-x} \int_{0}^{+\infty}  y e^ {-y}\,\mathrm{d}y
            \\
            \\
            & = &\displaystyle x^2 e^{-x}  \lim_{A \to \infty}  \int_{0}^{A} ye^{-y} \,\mathrm{d}y
            \\
            \\
            && \mbox{Ahora tenemos que $\displaystyle \int ye^{-y} dy = - \left(y+1\right) e^{-y}$}
            \\
            && \mbox{Por lo que si sustiyuimos tenemos : }
            \\
            \\
            & = & \displaystyle x^2 e^{-x}  \lim_{A \to \infty}  \left[- \left(y+1\right) e^{-y}\right] \vert _{0}^{A}
            \\
            \\
            & = & \displaystyle x^2 e^{-x} \left( \lim_{A \to \infty} \left[- \left(A+1\right) e^{-A}\right] - \left[- \left(0+1\right) e^{0}\right]\right)  
            \\
            \\
            && \mbox{tenemos que: $\lim_{A \to \infty}\left( \left(- \left(A+1\right) e^{-A}\right) + 1 \right)  = 1 $}
            \\
            \mbox{Por lo que : }&&
            \\
            \\
            \displaystyle \int_{0}^{+ \infty} x^2ye^{-\left(x+y\right)} \,\mathrm{d}y & = & \displaystyle x^2e^{-x}
        \end{array}
    \end{equation*}

    \begin{flushleft}
        Ahora tenemos que  $\displaystyle \int_{0}^{+\infty} \left(\int_{0}^{+\infty} x^2ye^{-\left(x+y\right)} \,\mathrm{d}y \right) \,\mathrm{d}x  = \displaystyle \int_{0}^{+\infty} \left(x^2e^{-x} \right) \,\mathrm{d}x  $  
    \end{flushleft}

    \begin{flushleft}
        Vamos ahora a calcular la integral : $\displaystyle \int_{0}^{+\infty} x^2e^{-x}  \,\mathrm{d}x$ , tenemos : 
    \end{flushleft}

    \begin{equation*}
        \begin{array}{rcl}
            \displaystyle \int_{0}^{+\infty} x^2e^{-x}  \,\mathrm{d}x & = & \displaystyle \lim_{A \to \infty} \int_{0}^{A} x^2e^{-x}  \,\mathrm{d}x  
            \\
            \\
            && \mbox{tenemos que: $\displaystyle \int_{}^{} x^2e^{-x}  \,\mathrm{d}x = \left(-x^2 -2x -2 \right)e^{-x}$}
            \\
            \\
            & = &\displaystyle \lim_{A \to \infty}  \left[\left(-x^2 -2x -2 \right)e^{-x}\right] \vert_{0}^{A}
            \\
            \\
            & = &\displaystyle  \lim_{A \to \infty}  \left(\left(-A^2 -2A -2 \right)e^{-A}\right) + 2 
            \\
            \\
            && \mbox{dado que $\displaystyle  \lim_{A \to \infty}  \left(\left(-A^2 -2A -2 \right)e^{-A}\right) = 0  $}
            \\
            \\
            \mbox{Entonces no queda : } &&
            \\
            \\
            \displaystyle \int_{0}^{+\infty} x^2e^{-x}  \,\mathrm{d}x  & = & 2 
        \end{array}
    \end{equation*}

    \begin{flushleft}
        Entonces tenemos que : 
    \end{flushleft}

    \begin{equation*}
        \fbox{$
            E\left[XY\right] = \displaystyle \int_{0}^{+\infty} \int_{0}^{+\infty} x^2ye^{-\left(x+y\right)} \,\mathrm{d}y  \,\mathrm{d}x = 2
        $}
    \end{equation*}

    \vspace{1cm}

    \begin{flushleft}
        Ahora para calcular los valores de $EX$ , $EY$ necesito primero calcular :
    \end{flushleft}

    \begin{equation*}
        \begin{array}{cp{1cm}c}
            \fbox{$
            f_X\left(x\right) =\displaystyle  \int_{-\infty}^{+\infty} f\left(x,y\right) \,\mathrm{d}y
            $}
            &
            &
            \fbox{$
            f_Y\left(y\right) =\displaystyle  \int_{-\infty}^{+\infty} f\left(x,y\right) \,\mathrm{d}x
            $}
        \end{array}
    \end{equation*}

    \begin{center}
        \begin{equation*}
            \begin{array}{c|c}
                \begin{array}{rcl}
                    f_{X}\left(x\right) & = & \displaystyle\int_{-\infty}^{+\infty}f{\left(X,Y\right)} dy 
                    \\
                    \\
                    f_{X}\left(x\right) & = & \displaystyle\int_{0}^{+\infty}xe^-\left(x+y\right)dy
                    \\
                    \\
                    f_{X}\left(x\right) & = & \displaystyle\int_{0}^{+\infty}xe^{-x}e^{-y}dy
                    \\
                    \\
                    f_{X}\left(x\right) & = & \displaystyle xe^{-x }\int_{0}^{+\infty}e^{-y}dy
                \end{array}
                &
                \begin{array}{l}
                    \mbox{Calculos adicionales :}
                    \\
                    \begin{array}{rcl}
                        \int_{0}^{+\infty}e^{-y}dy & = & \displaystyle\lim_{A\to\infty}\int_{0}^{A}e^{-y}dy
                        \\
                        \\
                        \int_{0}^{+\infty}e^{-y}dy & = & \displaystyle\lim_{A\to\infty}[-e^{-y}]\vert_{0}^{A}
                        \\
                        \\
                        \int_{0}^{+\infty}e^{-y}dy & = & -e^{-A} + e ^{-0}
                        \\
                        \\
                        \int_{0}^{+\infty}e^{-y}dy & = & 0 + 1
                        \\
                        \\
                        \int_{0}^{+\infty}e^{-y}dy & = & 1
                    \end{array}
                \end{array}
            \end{array}
        \end{equation*}
    \end{center}

    \begin{flushleft}
        Por lo tanto :  \fbox{$f_{X}(x) = xe^{-x}$}
    \end{flushleft}
   

    \vspace{0.5cm }

    \begin{flushleft}
        Ahora calculamos $f_{Y}(y)$
    \end{flushleft}
   
    \begin{center}
        \begin{equation*}
            \begin{array}{c|c}
                \begin{array}{rcl}
                    f_{Y}(Y) & = & \displaystyle\int_{-\infty}^{+\infty}f{\left(X,Y\right)} dy 
                    \\
                    \\
                    f_{Y}(Y) & = & \displaystyle\int_{0}^{+\infty}xe^-\left(x+y\right)dy
                    \\
                    \\
                    f_{Y}(Y) & = & \displaystyle\int_{0}^{+\infty}xe^{-x}e^{-y}dy
                    \\
                    \\
                    f_{Y}(Y) & = & \displaystyle e^{-y }\int_{0}^{+\infty}xe^{-x}dy
                    \\
                    \\
                    f_{Y}(Y) & = & e^{-y}
                \end{array}
                &
                \begin{array}{l}
                    \mbox{Calculos adicionales :}
                    \\
                    \begin{array}{rcl}
                        \\
                        \int_{}^{} xe^{-x} \,\mathrm{d}x  & = & -\left(x+1\right)e^{-x}
                        \\
                        \\
                        \int_{0}^{+\infty}xe^{-x}dx & = & \displaystyle\lim_{A\to\infty}\int_{0}^{A}xe^{-x}dx
                        \\
                        \\
                        & = & \displaystyle\lim_{A\to\infty}\left[-\left(x+1\right)e^{-x}\right]\vert_{0}^{A}
                        \\
                        \\
                        & = & \displaystyle\lim_{A\to\infty} \left[-\left(A+1\right)e^{-A}\right] +1 
                        \\
                        \\
                        \int_{0}^{+\infty}xe^{-x}dx & = & 1
                    \end{array}
                \end{array}
            \end{array}
        \end{equation*}
    \end{center}
    
    \begin{flushleft}
        Por lo tanto :  \fbox{$f_{Y}(Y) = e^{-y}$}
    \end{flushleft}
   
    \begin{flushleft}
        Vamos a calcular  $EX$   y $EY$
    \end{flushleft}
         
    \begin{equation*}
        \begin{array}{rcccccl}
            EX & = & \displaystyle \int_{0}^{-\infty}x f_{X}(X)dx & = & \displaystyle \int_{0}^{+\infty}x^{2}e^{-x}dx & = & 2
            \\
            \\
            EY & = &\displaystyle \int_{0}^{-\infty}y f_{Y}(Y)dy & = & \displaystyle \int_{0}^{+\infty}ye^{-y}dy & = & 1
        \end{array}
    \end{equation*}

    \begin{flushleft}
        Tenemos entonces que : \fbox{$EX = 2$}\hspace{1cm} \fbox{$EY = 1$}
    \end{flushleft}

    \vspace{0.5cm} 

    \begin{flushleft}
        Vamos ahora a calcular $EX^2 $ y $EY^2$  
    \end{flushleft}

    \begin{equation*}
        \fbox{$
            EX^2 = \displaystyle \int_{-\infty}^{+\infty}  x^2 f_{X}\left(x\right) \,\mathrm{d}x  = \displaystyle \int_{0}^{+\infty}  x^3 e^{-x} \,\mathrm{d}x
        $}
    \end{equation*}

    \begin{flushleft}
        Vamos ahora a calcular $\int_{0}^{+\infty}  x^3 e^{-x} \,\mathrm{d}x$ , tenemos : 
    \end{flushleft}
    \begin{equation*}
        \begin{array}{rcl}
            \displaystyle \int_{0}^{+\infty}  x^3 e^{-x} \,\mathrm{d}x & = & \displaystyle \lim_{A \to \infty}  \int_{0}^{A}  x^3 e^{-x} \,\mathrm{d}x
            \\
            \\
            && \mbox{tenemos : $ \int_{}^{}  x^3 e^{-x} \,\mathrm{d}x = - x^3e^{-x} - 3 \left(\left(x^2+ 2x + 2\right)e^{-x}\right)$} 
            \\
            && \mbox{por lo que nos queda que :}
            \\
            \\
            & = &\displaystyle \lim_{A \to \infty}  \left[- x^3e^{-x} - 3 \left(\left(x^2+ 2x + 2\right)e^{-x}\right)\right] \vert_{0}^{A}
            \\
            \\
            & = & \displaystyle \lim_{A \to \infty}  \left(- A^3e^{-A} - 3 \left(\left(A^2+ 2A + 2\right)e^{-A}\right)\right) + 6 
            \\
            \\
            && \mbox{como : $\lim_{A \to \infty}  \left(- A^3e^{-A} - 3 \left(\left(A^2+ 2A + 2\right)e^{-A}\right)\right) = 0 $}
            \\
            \\
            \mbox{Entonces nos queda que : }&&
            \\
            \\
            \displaystyle \int_{0}^{+\infty}  x^3 e^{-x} \,\mathrm{d}x & = & 6
        \end{array}
    \end{equation*}

    \begin{flushleft}
        Entonces tenemos el valor de $EX^2$
    \end{flushleft}

    \begin{equation*}
        \fbox{$
            EX^2 = \displaystyle \int_{0}^{+\infty}  x^3 e^{-x} \,\mathrm{d}x  =  6
        $}
    \end{equation*}

    \begin{flushleft}
        Ahora vamos a calcular $EY^2$
    \end{flushleft}

    \begin{equation*}
        \fbox{$
            EY^2 = \displaystyle \int_{-\infty}^{+\infty}  y^2 f_{Y}\left(y\right) \,\mathrm{d}y  = \displaystyle \int_{0}^{+\infty}  y^2 e^{-y} \,\mathrm{d}y
        $}
    \end{equation*}

    \begin{flushleft}
        tenemos ahora lo siguiente : 
    \end{flushleft}

    \begin{equation*}
        \begin{array}{rcl}
            \displaystyle \int_{0}^{+\infty}  y^2 e^{-y} \,\mathrm{d}y & = &\displaystyle \lim_{A \to \infty} \displaystyle \int_{0}^{A}  y^2 e^{-y} \,\mathrm{d}y
            \\
            \\
            && \mbox{sabemos que : $\displaystyle \int_{}^{}  y^2 e^{-y} \,\mathrm{d}y$}  =  - \left(y^2 + 2y + 2 \right) e^{-y}
            \\
            \\
            & = & \displaystyle \lim_{A \to \infty} \left[ - \left(y^2 + 2y + 2 \right) e^{-y}\right] \vert_{0}^{A}
            \\
            \\
            & = &  \displaystyle \lim_{A \to \infty} \left( - \left(A^2 + 2A + 2 \right) e^{-A}\right) + 2 
            \\
            \\
            && \mbox{tenemos que : $ \displaystyle \lim_{A \to \infty} \left( - \left(A^2 + 2A + 2 \right) e^{-A}\right) = 0 $} 
            \\
            \\
            \mbox{Por lo que tenemos que : }&&
            \\
            \\
            \displaystyle \int_{0}^{+\infty}  y^2 e^{-y} \,\mathrm{d}y & = & 2
        \end{array}
    \end{equation*}

    \begin{flushleft}
        Ahora tenemos tambien $EY^2$ :
    \end{flushleft}

    \begin{equation*}
        \fbox{$
            EY^2 = \displaystyle \int_{0}^{+\infty}  y^2 e^{-y} \,\mathrm{d}y  =  2
        $}
    \end{equation*}

    \vspace{1cm} 

    \begin{flushleft}
        Podemos ahora calcular la covarianza y la correlaci\'on 
    \end{flushleft}

    \begin{equation*}
        \fbox{$
            cov \left(X,Y\right) = E\left[XY\right] - EX \hspace{1mm}EY
        $}
    \end{equation*}

    \begin{flushleft}
        como tenemos que $E\left[XY\right] = 2 , EX =2  , EY=1 $ podemos sustituir y calcular . Por lo que obtenemos que : 
    \end{flushleft}

    \begin{equation*}
        \fbox{$
            cov \left(X,Y\right) = 0 
        $}
    \end{equation*}

    \begin{flushleft}
        Ahora para calcular la correlaci\'on necesitamos los valores de $V\left(X\right)$ y $V\left(Y\right)$
    \end{flushleft}

    \begin{equation*}
        \fbox{$
            V\left(X\right)  =  EX^2 - \left(EX\right)^2
        $}
    \end{equation*}

    \begin{flushleft}
        Ahora si sustituimos y calculamos nos queda: 
    \end{flushleft}
    \begin{equation*}
        \begin{array}{rcl}
            V\left(X\right) & = & EX^2 - \left(EX\right)^2
            \\ 
            \\
                            & = & 6 - \left(2\right)^2 = 6-4 
            \\
            \\
                            & = & 2
            \\
            \\
            V\left(Y\right) & = & EY^2  - \left(EY\right)^2
            \\
            \\
                            & = & 2 - \left(1\right)^2
            \\
            \\
                            & = & 1
        \end{array}
    \end{equation*}
    

    \begin{flushleft}
        Ahora para calcular la correlaci\'on tenemos : 
    \end{flushleft}

    \begin{equation*}
        \fbox{$
            \rho \left(X,Y\right) = \displaystyle \frac{cov \left(X,Y\right)}{\sqrt{V\left(X\right)V\left(Y\right)}}
        $}    
    \end{equation*}

    \begin{flushleft}
        Si sustituimos y calculamos tenemos : 
    \end{flushleft}

    \begin{equation*}
        \rho \left(X,Y\right) = \displaystyle \frac{cov \left(X,Y\right)}{\sqrt{V\left(X\right)V\left(Y\right)}} = \frac{0}{\sqrt{2}} = 0 
    \end{equation*}






    \section*{Ejercicio 4 } 

    \begin{flushleft}
        Sea $\left(X,Y\right)$  un vector aleatorio con funcion de densidad conjunta 
    \end{flushleft}

    \begin{equation*}
        f\left(x,y\right) = \begin{cases}
            \frac{1}{2x^2} & \mbox{si $ | x | < 1 , 0 \leq y \leq x^2$}
            \\
            0 & \mbox{en otro caso}
        \end{cases}
    \end{equation*}

    \subsection*{a-) Halle las funciones de densidad marginal de $X$ y de $Y$}

    \begin{flushleft}
        Primero vamos a calcular $f_{X}\left(x\right)$ 
    \end{flushleft}

    \begin{equation*}
        \begin{array}{rcl}
            \displaystyle f_{X}\left(x\right) & = & \displaystyle \int_{-\infty}^{+ \infty} f \left(x,y\right) \,\mathrm{d}y
            \\
            \\
            & =  & \displaystyle \int_{0}^{x^2} \frac{1}{2x^2} \,\mathrm{d}y   
            \\
            \\
            & = & \displaystyle \frac{1}{2x^2} \left[y\right] \vert_{0}^{x^2}
            \\
            \\
            & = & \displaystyle \frac{x^2}{2x^2}
            \\
            \\
            \displaystyle f_{X}\left(x\right) & = & \frac{1}{2} 
        \end{array}
    \end{equation*}

    \begin{flushleft}
        Ahora vamos a calcular $f_{Y} \left(y\right)$
    \end{flushleft}

    

    \begin{center}
         {\bf Aqui hay que rectificar el calculo de $f_{Y}\left(y\right)$} 
    \end{center}


    \subsection*{b) Calcule  $P\left(Y > 0,25 \right) $y $P \left(X > −0.5, Y < 0.64\right)$.}
    
    \vspace{0.5cm}
    \begin{flushleft}
        Vamos primero a calcular $P \left(Y > \frac{1}{4}\right)$ , tenemos : 
    \end{flushleft}

    \begin{equation*}
        P\left(Y>\frac{1}{4}\right) = \displaystyle \int_{-1}^{- \frac{1}{2}} \int_{\frac{1}{4}}^{x^2} \frac{1}{2x^2} \,\mathrm{d}y  \,\mathrm{d}x  + \int_{1}^{2} \int_{\frac{1}{4}}^{x^2} \frac{1}{2x^2} \,\mathrm{d}y  \,\mathrm{d}x 
    \end{equation*}

    \begin{flushleft}
        Vamos a calcular $\displaystyle \int_{-1}^{- \frac{1}{2}} \int_{\frac{1}{4}}^{x^2} \frac{1}{2x^2} \,\mathrm{d}y  \,\mathrm{d}x $
    \end{flushleft}

    \begin{equation*}
        \begin{array}{rcl}
            \displaystyle \int_{-1}^{- \frac{1}{2}} \int_{\frac{1}{4}}^{x^2} \frac{1}{2x^2} \,\mathrm{d}y  \,\mathrm{d}x  & = & \displaystyle  \frac{1}{2} \int_{-1}^{- \frac{1}{2}} \left(\int_{\frac{1}{4}}^{x^2}x^{-2}\,\mathrm{d}y \right) \,\mathrm{d}x 
            \\
            \\
            && \mbox{tenemos que: $\displaystyle \int_{\frac{1}{4}}^{x^2}x^{-2}\,\mathrm{d}y  = 1 - \frac{1}{4} x^{-2}$} 
            \\
            && \mbox{por lo tanto : }
            \\
            \\
            & = & \displaystyle  \frac{1}{2} \int_{-1}^{- \frac{1}{2}} \left(1 - \frac{1}{4} x^{-2}\right) \,\mathrm{d}x 
            \\
            \\
            && \mbox{tenemos : $\displaystyle  \int_{-1}^{- \frac{1}{2}} \left(1 - \frac{1}{4} x^{-2}\right) \,\mathrm{d}x = \frac{1}{4} $}
            \\
            \mbox{Por lo tanto: }
            \\
            \\
            \displaystyle \int_{-1}^{- \frac{1}{2}} \int_{\frac{1}{4}}^{x^2} \frac{1}{2x^2} \,\mathrm{d}y  \,\mathrm{d}x & = & \displaystyle \left(\frac{1}{2} \right) \left(\frac{1}{4}\right) = \frac{1}{8}
        \end{array}
    \end{equation*}

    \begin{flushleft}
        vamos ahora a calcular : $\displaystyle \int_{1}^{2} \int_{\frac{1}{4}}^{x^2} \frac{1}{2x^2} \,\mathrm{d}y  \,\mathrm{d}x$
    \end{flushleft}


    \begin{equation*}
        \begin{array}{rcl}
            \displaystyle \int_{\frac{1}{2}}^{1} \int_{\frac{1}{4}}^{x^2} \frac{1}{2x^2} \,\mathrm{d}y  \,\mathrm{d}x  & = & \displaystyle  \frac{1}{2} \int_{\frac{1}{2}}^{1} \left(\int_{\frac{1}{4}}^{x^2}x^{-2}\,\mathrm{d}y \right) \,\mathrm{d}x 
            \\
            \\
            && \mbox{tenemos que: $\displaystyle \int_{\frac{1}{4}}^{x^2}x^{-2}\,\mathrm{d}y  = 1 - \frac{1}{4} x^{-2}$} 
            \\
            && \mbox{por lo tanto : }
            \\
            \\
            & = & \displaystyle  \frac{1}{2} \int_{\frac{1}{2}}^{1} \left(1 - \frac{1}{4} x^{-2}\right) \,\mathrm{d}x 
            \\
            \\
            && \mbox{tenemos : $\displaystyle  \int_{\frac{1}{2}}^{1} \left(1 - \frac{1}{4} x^{-2}\right) \,\mathrm{d}x = \frac{1}{4} $}
            \\
            \mbox{Por lo tanto: }
            \\
            \\
            \displaystyle \int_{\frac{1}{2}}^{1} \int_{\frac{1}{4}}^{x^2} \frac{1}{2x^2} \,\mathrm{d}y  \,\mathrm{d}x & = & \displaystyle \left(\frac{1}{2} \right) \left(\frac{1}{4}\right) = \frac{1}{8}
        \end{array}
    \end{equation*}

    \begin{flushleft}
        Podemos concluir entonces que : 
    \end{flushleft}

    \begin{equation*}
        \begin{array}{rcl}
            P\left(Y>\frac{1}{4}\right) &= &  \displaystyle \int_{-1}^{- \frac{1}{2}} \int_{\frac{1}{4}}^{x^2} \frac{1}{2x^2} \,\mathrm{d}y  \,\mathrm{d}x  + \int_{1}^{2} \int_{\frac{1}{4}}^{x^2} \frac{1}{2x^2} \,\mathrm{d}y  \,\mathrm{d}x         
            \\
            \\
            P\left(Y>\frac{1}{4}\right) &= & \displaystyle  \frac{1}{8} + \frac{1}{8} = \frac{1}{4}
        \end{array}
    \end{equation*}

    \vspace{1cm}

    \begin{flushleft}
        Vamos ahora a calcular $P\left(X> - 0.5 , Y < 0.64\right)$ , tenemos :    
    \end{flushleft}

    \begin{equation*}
        P\left(X> - \frac{1}{2} , Y < 0.64\right) = \displaystyle \int_{-\frac{1}{2}}^{\frac{4}{5}} \int_{0}^{x^2} \frac{1}{2x^2} \,\mathrm{d}y  \,\mathrm{d}x  + \displaystyle \int_{\frac{4}{5}}^{1} \int_{0}^{0.64} \frac{1}{2x^2} \,\mathrm{d}y \,\mathrm{d}x
    \end{equation*}

    \begin{equation*}
        \begin{array}{rcl}
            \displaystyle \int_{-\frac{1}{2}}^{\frac{4}{5}} \int_{0}^{x^2} \frac{1}{2x^2} \,\mathrm{d}y  \,\mathrm{d}x  & = & \displaystyle \frac{1}{2} \int_{-\frac{1}{2}}^{\frac{4}{5}} \left(\int_{0}^{x^2} x^{-2} \,\mathrm{d}y \right)  \,\mathrm{d}x 
            \\
            \\
            && \mbox{tenemos que  $\displaystyle \int_{0}^{x^2} x^{-2} \,\mathrm{d}y  = 1 $}
            \\
            && \mbox{por lo tanto } 
            \\
            \\
            \displaystyle \int_{-\frac{1}{2}}^{\frac{4}{5}} \int_{0}^{x^2} \frac{1}{2x^2} \,\mathrm{d}y  \,\mathrm{d}x  & = & \displaystyle \frac{1}{2} \int_{-\frac{1}{2}}^{\frac{4}{5}}   \,\mathrm{d}x  
            \\
            \\
            & = & \displaystyle \frac{1}{2} \left(\left[x\right] \vert_{-\frac{1}{2}}^{\frac{4}{5}} \right)
            \\
            \\
            & = & \displaystyle \frac{1}{2} \left(\frac{4}{5} + \frac{1}{2}\right) = \left(\frac{1}{2}\right) \left(\frac{13}{10}\right) 
            \\
            \\
            \displaystyle \int_{-\frac{1}{2}}^{\frac{4}{5}} \int_{0}^{x^2} \frac{1}{2x^2} \,\mathrm{d}y  \,\mathrm{d}x & = & \displaystyle \frac{13}{20} 
        \end{array}
    \end{equation*}

    \hspace{1cm}

    \begin{flushleft}
        ahora calculamos la $ \displaystyle \int_{\frac{4}{5}}^{1} \int_{0}^{0.64} \frac{1}{2x^2} \,\mathrm{d}y \,\mathrm{d}x $
    \end{flushleft}

    \begin{equation*}
        \begin{array}{rcl}
            \displaystyle \int_{\frac{4}{5}}^{1} \int_{0}^{0.64} \frac{1}{2x^2} \,\mathrm{d}y \,\mathrm{d}x & = & \displaystyle  \frac{1}{2} \int_{\frac{4}{5}}^{1} \left(\int_{0}^{0.64} x^{-2} \,\mathrm{d}y \right) \,\mathrm{d}x 
            \\
            \\
            && \mbox{tenemos que $ \displaystyle \int_{0}^{0.64} x^{-2} \,\mathrm{d}y = \frac{64}{100}x^{-2} $}
            \\
            && \mbox{por lo tanto} 
            \\
            \\
            \displaystyle \int_{\frac{4}{5}}^{1} \int_{0}^{0.64} \frac{1}{2x^2} \,\mathrm{d}y \,\mathrm{d}x & = & \displaystyle  \frac{1}{2}\displaystyle \int_{\frac{4}{5}}^{1} \left(\frac{64}{100} x^{-2}\right) \,\mathrm{d}x
            \\
            \\
            & = & \displaystyle \left(\frac{1}{2}\right) \left(\frac{64}{100}\right) \int_{\frac{4}{5}}^{1} x^{-2} \,\mathrm{d}x 
            \\
            \\
            && \mbox{tenemos que $ \displaystyle \int_{\frac{4}{5}}^{1} x^{-2} \,\mathrm{d}x$} = \frac{1}{4}
            \\
            && \mbox{por lo tanto } 
            \\
            \\
            & = & \displaystyle \left(\frac{1}{2}\right) \left(\frac{64}{100}\right) \left(\frac{1}{4}\right) = \left(\frac{1}{2}\right) \left(\frac{16}{100}\right)
            \\
            \\
            \displaystyle \int_{\frac{4}{5}}^{1} \int_{0}^{0.64} \frac{1}{2x^2} \,\mathrm{d}y \,\mathrm{d}x & = & \displaystyle \frac{8}{100} = 0.08
        \end{array}
    \end{equation*}

    \begin{flushleft}
        Por lo que podemos concluir que : 
    \end{flushleft}

    \begin{equation*}
        P\left(X> - \frac{1}{2} , Y < 0.64\right) = \frac{13}{20} + \frac{8}{100} = \frac{73}{100} = 0.73
    \end{equation*}


    \vspace{2cm} 
    \section*{Ejercicio 5 }

    \subsection*{Determinar el valor de la constante $c$}

    \begin{flushleft}
        Sabemos que $ \displaystyle \int_{-\infty}^{+\infty} \int_{-\infty}^{+\infty} f\left(x,y\right) \,\mathrm{d}x  \,\mathrm{d}y   =1 $
    \end{flushleft}

    \begin{flushleft}
        Para nuestro caso tenemos que $\displaystyle \int_{0}^{2} \int_{0}^{x} c\left(x+y \right) \,\mathrm{d}y \,\mathrm{d}x   =1 $ , entonces : 
    \end{flushleft}

    \begin{equation*}
        \begin{array}{rcl}
            \displaystyle \int_{0}^{2} \int_{0}^{x} c\left(x+y \right) \,\mathrm{d}y \,\mathrm{d}x  & = & \displaystyle c  \int_{0}^{2} \left( \int_{0}^{x} \left(x+y \right) \,\mathrm{d}y \right)\,\mathrm{d}x 
            \\
            \\
            && \mbox{pero la $ \displaystyle  \int_{0}^{x} \left(x+y \right) \,\mathrm{d}y = x^2 + \frac{x^2}{2} $}
            \\
            && \mbox{por lo que tenemos : }
            \\
            \\
            & = &  \displaystyle c  \int_{0}^{2} \left(x^2 + \frac{x^2}{2}\right)\,\mathrm{d}x 
            \\
            \\
            && \mbox{tenemos que $ \displaystyle \int_{0}^{2} \left(x^2 + \frac{x^2}{2}\right)\,\mathrm{d}x  = \frac{10}{3}$ }
            \\
            \mbox{Por lo que tenemos  :}&&
            \\
            \\
            \displaystyle \int_{0}^{2} \int_{0}^{x} c\left(x+y \right) \,\mathrm{d}y \,\mathrm{d}x  & = & \displaystyle c \left(\frac{10}{3}\right)
            \\
            \\
            \mbox{Si sustituimos aqui }&&
            \\
            \\
            1 & = & \displaystyle c \left(\frac{10}{3}\right)
        \end{array}
    \end{equation*}

    \begin{flushleft}
        Por lo que nos queda que : $ \displaystyle c = \frac{3}{10}$
    \end{flushleft}

\end{document}