\documentclass[12pt]{article}

\usepackage{amsmath}

\begin{document}
	\title{Respuesta de la Clase Pr\'actica 10}
	\author{Daniel De La Cruz Prieto}
	\date{\today}
	
    \maketitle
    
    \section*{Ejercicio 1 } 

    \subsection*{ 1-a) La funci\'on de densidad de un vector aleatorio continuo es :}

    \begin{equation*}
        f\left(x,y\right) = \begin{cases}
    
        xe^{-\left(X+Y\right)} & \mbox{$x\leq0$ y $y\leq 0$}
        \\
               0             & \mbox{en otro caso}
    
       \end{cases}
    \end{equation*}

    \begin{flushleft}
        Tenemos $F_{\left(X\right)}$ la podemos dobtener de la siguiente expresi\'on
    \end{flushleft}

   	\begin{equation*} 
	    \fbox{$
	   	F_{X} =  \displaystyle \lim_{y \to + \infty} F\left(x,y\right) = \displaystyle  F_{\left(X,Y\right)} \left(x,+\infty\right) = \int_{0}^{x}\int_{0}^\infty f\left(u,y\right) dy du	   	
	   	$}
    \end{equation*} 

	\begin{equation*} 
	    \begin{array}{rcl}
	        \to\int_{0}^{x} \int_{0}^{+\infty} f \left(u,y\right) dy du & = &  \int_{0}^{x} \left(\int_{o}^{+ \infty}xe^{-\left(x + y\right)} dy\right)du
	        \\
	        \\
	        \to \int_{0}^{+ \infty} xe^{-\left(x+y\right)}dy & = & xe ^{-\left(x\right)} \int_{0}^{+ \infty} e^{-\left(y\right)}dy = \displaystyle xe^{-\left(x\right)} * \lim_{A\to + \infty} \int_{0}^{A} e^{-\left(y\right)} dy
	        \\
	        \\
	        & = &\displaystyle xe^{-\left(x\right)}  \lim_{A\to+\infty} \left[ -e^{\left(-y\right)}\right]\vert_{0}^{A}   = \displaystyle xe^{-\left(x\right)}* \lim_{A\to+\infty} \left[ -e^{-A} + e^0\right]
	        \\
	        \\
            & = & xe^{-\left(x\right)}
            \\
	    \end{array}
    \end{equation*}        
  
    Entonces :

    \begin{equation*}
        \int_{o}^{+\infty} ue^{-u}du = \int_{0}^{x}ue^{-u} = \int_{o}^{x}   ue^{-u} du
    \end{equation*}
    
    Luego teniendo que :
    
    \begin{equation*}
       \fbox{$
        \int xe^{-x}d = -\left(x+1\right)e^{-x}
        $}
    \end{equation*}

    \begin{equation*}
        \begin{array}{rcl}
            \int_{o}^{x} ue^{-u}du & = & \displaystyle  [-\left(u+1\right)e^{-u}]\vert_{o}^{x}
            \\
            \\
            \int_{o}^{x} ue^{-u}du & = & \displaystyle -\left(x+1\right)e^{-x} - [-\left(0+1\right)e^{-0}]
            \\
            \\
            & = & -\left(x+1\right)e^{-x}+1
            \\
            \\
            \int_{o}^{x} ue^{-u}du & = & 1 -\left(x+1\right)e^{-x} 
            \\
        \end{array}
    \end{equation*}

    Por lo tanto :

    \begin{equation*}
       \fbox{$F_{X} = 1 - \left(x+1\right)e^{-x}$}
    \end{equation*}

    Dada la ecuaci\'on siguinte para calcular $F_{Y}$ :

    \begin{equation*} 
        \fbox{$
            F_{Y} = \displaystyle\lim_{x\to+\infty} F_{\left(x,y\right)}\left(+\infty,x\right)=\int_{-\infty}^{y} \int_{-\infty}^{+\infty} f\left(x,u\right) dx du
        $}
    \end{equation*} 

    \begin{equation*}
        \begin{array}{rcl}
            \int_{0}^{y}\int_{0}^{+\infty}xe^{-\left(x+u\right)}dx du & = & 
            \int_{0}^{y} \left(\int_{0}^{+\infty}xe^{-\left(x+u\right)}dx\right)du 
            \\
            \\
            \int_{0}^{y}\int_{0}^{+\infty}xe^{-\left(x+u\right)}dx du & = & \int_{0}^{+\infty}xe^{-x}e^{-u}dx = e^{-u}\int_{0}^{+\infty}xe^{-x}dx 
            \\
            \\
            \to\int_{0}^{+\infty}xe^{-\left(x+u\right)dx du} & = & \int_{0}^{+\infty}xe^{-x}e^{-u}dx = e^{-u}\int_{0}^{+\infty}xe^{-x}dx
            \\
        \end{array}
    \end{equation*}
    Tenamos que :
      \begin{equation*}
         \fbox{$\int xe^{-x}dx = -\left(x+1\right)e^{-x}$}
      \end{equation*}
    Por lo tanto :
    \begin{equation*}
        \begin{array}{rcl}
            \to\int_{0}^{+\infty}xe^{-x} dx & = & \displaystyle \lim_{A\to+\infty}\int_{0}^{A}xe^{-x}dx
            \\
            \\
            \to\int_{0}^{+\infty}xe^{-x} dx & = &  \displaystyle \lim_{A\to + \infty}[-\left(x+1\right)e^{-x}] \vert_{0}^{A} 
            \\
            \\
            \to\int_{0}^{+\infty}xe^{-x} dx & = &  \displaystyle\lim{A\to+\infty} -\left(A+1\right)e^{-A}+1
            \\
            \\
            \to\int_{0}^{+\infty}xe^{-x} dx & = &  \displaystyle \lim_{A\to+\infty}-Ae^{-A}-e^{-A}+1
            \\
            \\
            \to\int_{0}^{+\infty}xe^{-x} dx & = &  \displaystyle\lim_{A\to+\infty}-Ae^{-A}-\lim_{A\to+\infty}e^{-A}+1
            \\
            \\
            \to\int_{0}^{+\infty}xe^{-x} dx & = &  1
        \end{array}
    \end{equation*}

    Luego comprobando que:

    \begin{equation*}
        \begin{array}{rcl}
            \int_{0}^{y}\left(\int_{0}^{+\infty}xe^{-\left(x+u\right)}dx\right)du & = & \int_{0}^{y} e^{-u }du 
            \\
            \\
            \int_{0}^{y}\left(\int_{0}^{+\infty}xe^{-\left(x+u\right)}dx\right)du & = & -e^{-4}\vert_{o}^{y}
            \\
            \\
            \int_{0}^{y}\left(\int_{0}^{+\infty}xe^{-\left(x+u\right)}dx\right)du & = & -e^{-y}+e^{-0}
            \\
            \\
            \int_{0}^{y}\left(\int_{0}^{+\infty}xe^{-\left(x+u\right)}dx\right)du & = & 1 -e^{-y}
        \end{array}
    \end{equation*}

    Finalmente tenemos que :

    \begin{equation*}
        \fbox{$
        F_{Y}= 1 - e^{-y}
        $}
    \end{equation*}


    \subsection*{1-b) Determinar si las variables $X$ y $Y$ son independientes y $\rho \left(X,Y\right)$}
    
    \begin{flushleft}
        Vamos a comprobar que $F_{X,Y}\left(x,y\right) = F_Y\left(y\right) F_X\left(x\right)$
    \end{flushleft}

    \begin{flushleft}
        Entonces tenemos que : 
    \end{flushleft}

    \begin{equation*}
        \begin{array}{rcl}
            \displaystyle F_{X,Y} \left(x,y\right) & = & \displaystyle \int_{- \infty}^{y} \int_{-\infty}^{x} f \left(u,v \right) \,\mathrm{d}u  \,\mathrm{d}v
            \\
            \\
            & = & \displaystyle \int_{0}^{y}  \int_{0}^{x} u e^{-\left(u+v\right)} \,\mathrm{d}u \,\mathrm{d}v 
            \\
            \\
            & = & \displaystyle \int_{0}^{y} e^{-v} \int_{0}^{x} u e^{-u} \,\mathrm{d}u \,\mathrm{d}v
            \\
            \\
            &= & \displaystyle \int_{0}^{y} e^{-v} \,\mathrm{d}v \left(\frac{e^x - x -1 }{e^x}\right)  
            \\
            \\
            & = &\displaystyle \frac{e^y -1 }{e^y} \left(1 - \frac{x-1}{e^x}\right)
            \\
            \\
            & = & \displaystyle \left(1 - e^{-y}\right) \left(1- \frac{x-1}{e^x}\right)
            \\
            \\
            \displaystyle F_{X,Y} \left(x,y\right)& = & \displaystyle  F_Y\left(y\right) F_X\left(x\right)
        \end{array}
    \end{equation*}

    \begin{flushleft}
        Entonces las variables aleatorias $X$ y $Y$ son independientes   
    \end{flushleft}

    \vspace{1cm}


    \begin{flushleft}
        Para calcular $\rho \left(X,Y\right)$  tenemos que calcular primero : $E\left[XY\right]$ , $EX$, $EY$ , $EX^2$ y $EY^2$   
    \end{flushleft}
    

    \begin{flushleft}
        Primero vamos a calcular $E\left[XY\right]$ para eso tenemos: 
    \end{flushleft}

    \begin{equation*}
        \fbox{$
        E\left[XY\right] = \displaystyle \int_{- \infty}^{+ \infty} \int_{-\infty}^{+\infty} xy f\left(x,y\right) \,\mathrm{d}y  \,\mathrm{d}x  = \int_{0}^{+\infty} \int_{0}^{+\infty} x^2ye^{-\left(x+y\right)} \,\mathrm{d}y  \,\mathrm{d}x 
        $}
    \end{equation*}

    \begin{flushleft}
        Primero calculo $\int_{0}^{+ \infty} x^2ye^{-\left(x+y\right)} \,\mathrm{d}y $ 
    \end{flushleft}

    \begin{equation*}
        \begin{array}{rcl}
            \displaystyle \int_{0}^{+ \infty} x^2ye^{-\left(x+y\right)} \,\mathrm{d}y  &= & \displaystyle  x^2 e^{-x} \int_{0}^{+\infty}  y e^ {-y}\,\mathrm{d}y
            \\
            \\
            & = &\displaystyle x^2 e^{-x}  \lim_{A \to \infty}  \int_{0}^{A} ye^{-y} \,\mathrm{d}y
            \\
            \\
            && \mbox{Ahora tenemos que $\displaystyle \int ye^{-y} dy = - \left(y+1\right) e^{-y}$}
            \\
            && \mbox{Por lo que si sustiyuimos tenemos : }
            \\
            \\
            & = & \displaystyle x^2 e^{-x}  \lim_{A \to \infty}  \left[- \left(y+1\right) e^{-y}\right] \vert _{0}^{A}
            \\
            \\
            & = & \displaystyle x^2 e^{-x} \left( \lim_{A \to \infty} \left[- \left(A+1\right) e^{-A}\right] - \left[- \left(0+1\right) e^{0}\right]\right)  
            \\
            \\
            && \mbox{tenemos que: $\lim_{A \to \infty}\left( \left(- \left(A+1\right) e^{-A}\right) + 1 \right)  = 1 $}
            \\
            \mbox{Por lo que : }&&
            \\
            \\
            \displaystyle \int_{0}^{+ \infty} x^2ye^{-\left(x+y\right)} \,\mathrm{d}y & = & \displaystyle x^2e^{-x}
        \end{array}
    \end{equation*}

    \begin{flushleft}
        Ahora tenemos que  $\displaystyle \int_{0}^{+\infty} \left(\int_{0}^{+\infty} x^2ye^{-\left(x+y\right)} \,\mathrm{d}y \right) \,\mathrm{d}x  = \displaystyle \int_{0}^{+\infty} \left(x^2e^{-x} \right) \,\mathrm{d}x  $  
    \end{flushleft}

    \begin{flushleft}
        Vamos ahora a calcular la integral : $\displaystyle \int_{0}^{+\infty} x^2e^{-x}  \,\mathrm{d}x$ , tenemos : 
    \end{flushleft}

    \begin{equation*}
        \begin{array}{rcl}
            \displaystyle \int_{0}^{+\infty} x^2e^{-x}  \,\mathrm{d}x & = & \displaystyle \lim_{A \to \infty} \int_{0}^{A} x^2e^{-x}  \,\mathrm{d}x  
            \\
            \\
            && \mbox{tenemos que: $\displaystyle \int_{}^{} x^2e^{-x}  \,\mathrm{d}x = \left(-x^2 -2x -2 \right)e^{-x}$}
            \\
            \\
            & = &\displaystyle \lim_{A \to \infty}  \left[\left(-x^2 -2x -2 \right)e^{-x}\right] \vert_{0}^{A}
            \\
            \\
            & = &\displaystyle  \lim_{A \to \infty}  \left(\left(-A^2 -2A -2 \right)e^{-A}\right) + 2 
            \\
            \\
            && \mbox{dado que $\displaystyle  \lim_{A \to \infty}  \left(\left(-A^2 -2A -2 \right)e^{-A}\right) = 0  $}
            \\
            \\
            \mbox{Entonces no queda : } &&
            \\
            \\
            \displaystyle \int_{0}^{+\infty} x^2e^{-x}  \,\mathrm{d}x  & = & 2 
        \end{array}
    \end{equation*}

    \begin{flushleft}
        Entonces tenemos que : 
    \end{flushleft}

    \begin{equation*}
        \fbox{$
            E\left[XY\right] = \displaystyle \int_{0}^{+\infty} \int_{0}^{+\infty} x^2ye^{-\left(x+y\right)} \,\mathrm{d}y  \,\mathrm{d}x = 2
        $}
    \end{equation*}

    \vspace{1cm}

    \begin{flushleft}
        Ahora para calcular los valores de $EX$ , $EY$ necesito primero calcular :
    \end{flushleft}

    \begin{equation*}
        \begin{array}{cp{1cm}c}
            \fbox{$
            f_X\left(x\right) =\displaystyle  \int_{-\infty}^{+\infty} f\left(x,y\right) \,\mathrm{d}y
            $}
            &
            &
            \fbox{$
            f_Y\left(y\right) =\displaystyle  \int_{-\infty}^{+\infty} f\left(x,y\right) \,\mathrm{d}x
            $}
        \end{array}
    \end{equation*}

    \begin{center}
        \begin{equation*}
            \begin{array}{c|c}
                \begin{array}{rcl}
                    f_{X}\left(x\right) & = & \displaystyle\int_{-\infty}^{+\infty}f{\left(X,Y\right)} dy 
                    \\
                    \\
                    f_{X}\left(x\right) & = & \displaystyle\int_{0}^{+\infty}xe^-\left(x+y\right)dy
                    \\
                    \\
                    f_{X}\left(x\right) & = & \displaystyle\int_{0}^{+\infty}xe^{-x}e^{-y}dy
                    \\
                    \\
                    f_{X}\left(x\right) & = & \displaystyle xe^{-x }\int_{0}^{+\infty}e^{-y}dy
                \end{array}
                &
                \begin{array}{l}
                    \mbox{Calculos adicionales :}
                    \\
                    \begin{array}{rcl}
                        \int_{0}^{+\infty}e^{-y}dy & = & \displaystyle\lim_{A\to\infty}\int_{0}^{A}e^{-y}dy
                        \\
                        \\
                        \int_{0}^{+\infty}e^{-y}dy & = & \displaystyle\lim_{A\to\infty}[-e^{-y}]\vert_{0}^{A}
                        \\
                        \\
                        \int_{0}^{+\infty}e^{-y}dy & = & -e^{-A} + e ^{-0}
                        \\
                        \\
                        \int_{0}^{+\infty}e^{-y}dy & = & 0 + 1
                        \\
                        \\
                        \int_{0}^{+\infty}e^{-y}dy & = & 1
                    \end{array}
                \end{array}
            \end{array}
        \end{equation*}
    \end{center}

    \begin{flushleft}
        Por lo tanto :  \fbox{$f_{X}(x) = xe^{-x}$}
    \end{flushleft}
   

    \vspace{0.5cm }

    \begin{flushleft}
        Ahora calculamos $f_{Y}(y)$
    \end{flushleft}
   
    \begin{center}
        \begin{equation*}
            \begin{array}{c|c}
                \begin{array}{rcl}
                    f_{Y}(Y) & = & \displaystyle\int_{-\infty}^{+\infty}f{\left(X,Y\right)} dy 
                    \\
                    \\
                    f_{Y}(Y) & = & \displaystyle\int_{0}^{+\infty}xe^-\left(x+y\right)dy
                    \\
                    \\
                    f_{Y}(Y) & = & \displaystyle\int_{0}^{+\infty}xe^{-x}e^{-y}dy
                    \\
                    \\
                    f_{Y}(Y) & = & \displaystyle e^{-y }\int_{0}^{+\infty}xe^{-x}dy
                    \\
                    \\
                    f_{Y}(Y) & = & e^{-y}
                \end{array}
                &
                \begin{array}{l}
                    \mbox{Calculos adicionales :}
                    \\
                    \begin{array}{rcl}
                        \\
                        \int_{}^{} xe^{-x} \,\mathrm{d}x  & = & -\left(x+1\right)e^{-x}
                        \\
                        \\
                        \int_{0}^{+\infty}xe^{-x}dx & = & \displaystyle\lim_{A\to\infty}\int_{0}^{A}xe^{-x}dx
                        \\
                        \\
                        & = & \displaystyle\lim_{A\to\infty}\left[-\left(x+1\right)e^{-x}\right]\vert_{0}^{A}
                        \\
                        \\
                        & = & \displaystyle\lim_{A\to\infty} \left[-\left(A+1\right)e^{-A}\right] +1 
                        \\
                        \\
                        \int_{0}^{+\infty}xe^{-x}dx & = & 1
                    \end{array}
                \end{array}
            \end{array}
        \end{equation*}
    \end{center}
    
    \begin{flushleft}
        Por lo tanto :  \fbox{$f_{Y}(Y) = e^{-y}$}
    \end{flushleft}
   
    \begin{flushleft}
        Vamos a calcular  $EX$   y $EY$
    \end{flushleft}
         
    \begin{equation*}
        \begin{array}{rcccccl}
            EX & = & \displaystyle \int_{0}^{-\infty}x f_{X}(X)dx & = & \displaystyle \int_{0}^{+\infty}x^{2}e^{-x}dx & = & 2
            \\
            \\
            EY & = &\displaystyle \int_{0}^{-\infty}y f_{Y}(Y)dy & = & \displaystyle \int_{0}^{+\infty}ye^{-y}dy & = & 1
        \end{array}
    \end{equation*}

    \begin{flushleft}
        Tenemos entonces que : \fbox{$EX = 2$}\hspace{1cm} \fbox{$EY = 1$}
    \end{flushleft}

    \vspace{0.5cm} 

    \begin{flushleft}
        Vamos ahora a calcular $EX^2 $ y $EY^2$  
    \end{flushleft}

    \begin{equation*}
        \fbox{$
            EX^2 = \displaystyle \int_{-\infty}^{+\infty}  x^2 f_{X}\left(x\right) \,\mathrm{d}x  = \displaystyle \int_{0}^{+\infty}  x^3 e^{-x} \,\mathrm{d}x
        $}
    \end{equation*}

    \begin{flushleft}
        Vamos ahora a calcular $\int_{0}^{+\infty}  x^3 e^{-x} \,\mathrm{d}x$ , tenemos : 
    \end{flushleft}
    \begin{equation*}
        \begin{array}{rcl}
            \displaystyle \int_{0}^{+\infty}  x^3 e^{-x} \,\mathrm{d}x & = & \displaystyle \lim_{A \to \infty}  \int_{0}^{A}  x^3 e^{-x} \,\mathrm{d}x
            \\
            \\
            && \mbox{tenemos : $ \int_{}^{}  x^3 e^{-x} \,\mathrm{d}x = - x^3e^{-x} - 3 \left(\left(x^2+ 2x + 2\right)e^{-x}\right)$} 
            \\
            && \mbox{por lo que nos queda que :}
            \\
            \\
            & = &\displaystyle \lim_{A \to \infty}  \left[- x^3e^{-x} - 3 \left(\left(x^2+ 2x + 2\right)e^{-x}\right)\right] \vert_{0}^{A}
            \\
            \\
            & = & \displaystyle \lim_{A \to \infty}  \left(- A^3e^{-A} - 3 \left(\left(A^2+ 2A + 2\right)e^{-A}\right)\right) + 6 
            \\
            \\
            && \mbox{como : $\lim_{A \to \infty}  \left(- A^3e^{-A} - 3 \left(\left(A^2+ 2A + 2\right)e^{-A}\right)\right) = 0 $}
            \\
            \\
            \mbox{Entonces nos queda que : }&&
            \\
            \\
            \displaystyle \int_{0}^{+\infty}  x^3 e^{-x} \,\mathrm{d}x & = & 6
        \end{array}
    \end{equation*}

    \begin{flushleft}
        Entonces tenemos el valor de $EX^2$
    \end{flushleft}

    \begin{equation*}
        \fbox{$
            EX^2 = \displaystyle \int_{0}^{+\infty}  x^3 e^{-x} \,\mathrm{d}x  =  6
        $}
    \end{equation*}

    \begin{flushleft}
        Ahora vamos a calcular $EY^2$
    \end{flushleft}

    \begin{equation*}
        \fbox{$
            EY^2 = \displaystyle \int_{-\infty}^{+\infty}  y^2 f_{Y}\left(y\right) \,\mathrm{d}y  = \displaystyle \int_{0}^{+\infty}  y^2 e^{-y} \,\mathrm{d}y
        $}
    \end{equation*}

    \begin{flushleft}
        tenemos ahora lo siguiente : 
    \end{flushleft}

    \begin{equation*}
        \begin{array}{rcl}
            \displaystyle \int_{0}^{+\infty}  y^2 e^{-y} \,\mathrm{d}y & = &\displaystyle \lim_{A \to \infty} \displaystyle \int_{0}^{A}  y^2 e^{-y} \,\mathrm{d}y
            \\
            \\
            && \mbox{sabemos que : $\displaystyle \int_{}^{}  y^2 e^{-y} \,\mathrm{d}y$}  =  - \left(y^2 + 2y + 2 \right) e^{-y}
            \\
            \\
            & = & \displaystyle \lim_{A \to \infty} \left[ - \left(y^2 + 2y + 2 \right) e^{-y}\right] \vert_{0}^{A}
            \\
            \\
            & = &  \displaystyle \lim_{A \to \infty} \left( - \left(A^2 + 2A + 2 \right) e^{-A}\right) + 2 
            \\
            \\
            && \mbox{tenemos que : $ \displaystyle \lim_{A \to \infty} \left( - \left(A^2 + 2A + 2 \right) e^{-A}\right) = 0 $} 
            \\
            \\
            \mbox{Por lo que tenemos que : }&&
            \\
            \\
            \displaystyle \int_{0}^{+\infty}  y^2 e^{-y} \,\mathrm{d}y & = & 2
        \end{array}
    \end{equation*}

    \begin{flushleft}
        Ahora tenemos tambien $EY^2$ :
    \end{flushleft}

    \begin{equation*}
        \fbox{$
            EY^2 = \displaystyle \int_{0}^{+\infty}  y^2 e^{-y} \,\mathrm{d}y  =  2
        $}
    \end{equation*}

    \vspace{1cm} 

    \begin{flushleft}
        Podemos ahora calcular la covarianza y la correlaci\'on 
    \end{flushleft}

    \begin{equation*}
        \fbox{$
            cov \left(X,Y\right) = E\left[XY\right] - EX \hspace{1mm}EY
        $}
    \end{equation*}

    \begin{flushleft}
        como tenemos que $E\left[XY\right] = 2 , EX =2  , EY=1 $ podemos sustituir y calcular . Por lo que obtenemos que : 
    \end{flushleft}

    \begin{equation*}
        \fbox{$
            cov \left(X,Y\right) = 0 
        $}
    \end{equation*}

    \begin{flushleft}
        Ahora para calcular la correlaci\'on necesitamos los valores de $V\left(X\right)$ y $V\left(Y\right)$
    \end{flushleft}

    \begin{equation*}
        \fbox{$
            V\left(X\right)  =  EX^2 - \left(EX\right)^2
        $}
    \end{equation*}

    \begin{flushleft}
        Ahora si sustituimos y calculamos nos queda: 
    \end{flushleft}
    \begin{equation*}
        \begin{array}{rcl}
            V\left(X\right) & = & EX^2 - \left(EX\right)^2
            \\ 
            \\
                            & = & 6 - \left(2\right)^2 = 6-4 
            \\
            \\
                            & = & 2
            \\
            \\
            V\left(Y\right) & = & EY^2  - \left(EY\right)^2
            \\
            \\
                            & = & 2 - \left(1\right)^2
            \\
            \\
                            & = & 1
        \end{array}
    \end{equation*}
    

    \begin{flushleft}
        Ahora para calcular la correlaci\'on tenemos : 
    \end{flushleft}

    \begin{equation*}
        \fbox{$
            \rho \left(X,Y\right) = \displaystyle \frac{cov \left(X,Y\right)}{\sqrt{V\left(X\right)V\left(Y\right)}}
        $}    
    \end{equation*}

    \begin{flushleft}
        Si sustituimos y calculamos tenemos : 
    \end{flushleft}

    \begin{equation*}
        \rho \left(X,Y\right) = \displaystyle \frac{cov \left(X,Y\right)}{\sqrt{V\left(X\right)V\left(Y\right)}} = \frac{0}{\sqrt{2}} = 0 
    \end{equation*}


    \section*{Ejercicio 2 } 

    \begin{flushleft}
        Sea $\left(X,Y\right) $ un vector aleatorio continuo con funcion de densidad $ f\left(X,Y\right) = k $ , esta definida en la regi\'on limitada por las rectas $ x = 0 , y = 0 , x + y = 1 $.
    \end{flushleft}
    
    (Inciso 2-a)$ \to$ Encuentre el valor de K
     
    (Inciso 2-b)$ \to $ Calcule $  P\left(X>0,Y<\frac{1}{2}\right) $
    
    (Inciso 2-c)$ \to $ Calcule $ V(X) $
    \paragraph{Respuesta del (inciso 2-a)}  Sabemos que : 
    
    \begin{equation*}
        \fbox{$
            \displaystyle F_{\left(X,Y\right)} \left(+\infty,+\infty\right)  = \displaystyle \int_{-\infty}^{+\infty}\int_{-\infty}^{+\infty}f\left(X,Y\right)dy dx = 1
        $} 
    \end{equation*}

    \begin{flushleft}
        Como el \'area est\'a determinada por : $ 0\leq x \leq1  $ y $   0\leq y \leq1-x $ 
        Entonces podemos plantear que : 
    \end{flushleft}
         
    \begin{equation*}
        \begin{array}{rcl}
            \displaystyle \int_{0}^{1}\int_{0}^{1-x}K dy dx & = & \displaystyle\int_{0}^{1}\left(\int_{o}^{1-x}k dy\right)dx
            \\
            \\
            \displaystyle \int_{0}^{1-x}K dy & = & K\int_{0}^{1-x}dy
            \\
            \\
            & = & K[y]\vert_{0}^{1-x}
            \\
            \\
            & = & K[(1-x)] = K - Kx
            \\
            \\
            & = & K(1-x)
        \end{array}
    \end{equation*}

    \begin{center}
        \begin{equation*}
          \begin{array}{c|c}
                \begin{array}{rcl}
                    \displaystyle\int_{0}^{1}\int_{0}^{1-x} K dy dx & = & \displaystyle\int_{0}^{1} k\left(1-x\right)dx
                    \\
                    \\
                    & = & \displaystyle K\int_{0}^{1}\left(1-x\right)dx
                    \\
                    \\
                    & = &\displaystyle K\left[-\frac{\left(1-x\right)^2}{2}\right]\vert_{0}^{1}
                    \\
                    \\
                    & = &\displaystyle K\left[-\frac{\left(1-x\right)^2}{2} + \frac{\left(1-0\right)^2}{2}\right] 
                    \\
                    \\
                    & = &\displaystyle K\left[0 +\frac{1}{2}\right]
                    \\
                    \\
                    & = & K\frac{1}{2}
                    \\
                    \\
                     & = & \frac{K}{2}
              \end{array}
               &
              \begin{array}{c}
                   \mbox{Calculos Adicionales :}
                   \\
                   \begin{array}{rcl}
                          \\
                          \\
                          \displaystyle\int_{}^{} \left(1-x\right)dx & = & \displaystyle\int_{}^{} -u du
                          \\
                          \\
                          & = & -\int_{}^{}u du
                          \\
                          \\
                          & = & -\frac{u^2}{2}
                          \\
                          \\
                          & = & -\frac{\left(1-x\right)^2}{2} 
                   \end{array}
              \end{array}
            \end{array}
         \end{equation*}
    \end{center}

    \begin{flushleft}
        Entonces tenemos que : $\displaystyle \int_{0}^{1} \int_{0}^{1-x}K dy dx = \frac{K}{2} = 1$ 
        Por lo tanto : \fbox{$K = 2$}
    \end{flushleft}
     
     
    \subsection*{Respuesta del inciso 2-b}

    \begin{flushleft}
        Como $P\left(X\leq x,Y\leq y\right) = \int_{-\infty}^{x}\int_{-\infty}^{y}f\left(u,v\right)du dv$ en el dominio de la funci\'on tenemos que :
    \end{flushleft}
     
      
    \begin{equation*}
        \fbox{$ 
            \displaystyle \int_{0}^{y_2}\int_{0}^{1-y}K dy dx = \displaystyle \int_{0}^{\frac{1}{2}}\left(\int_{0}^{1-y}Kdx\right) dy 
        $}
    \end{equation*}

    \begin{equation*}
        \begin{array}{rcl}
            \displaystyle \int_{0}^{\frac{1}{2}} \left(\int_{0}^{1-y} K dx \right) dy & = &\displaystyle \int_{0}^{\frac{1}{2}}\left(\int_{0}^{1-y}2dx\right)dy
            \\
            \\
            & = & \displaystyle\int_{0}^{1-y}2 dx 
            \\
            \\
            & = & \displaystyle 2\int_{0}^{1-y}dx
            \\
            \\
            & = & \displaystyle2\left[x\right]\vert_{0}^{1-y}
            \\
            \\
            & = & \displaystyle 2\left(1-y\right)
        \end{array}
    \end{equation*}

    \begin{center}
        \begin{equation*}
            \begin{array}{c|c}
                \begin{array}{rcl}
                    \displaystyle \int_{0}^{\frac{1}{2}} 2\left(1-y\right) dy & = & \displaystyle 2\int_{0}^{\frac{1}{2}}\left(1-y\right)dy
                    \\
                    \\
                    \displaystyle \int_{0}^{\frac{1}{2}} 2\left(1-y\right) dy & = & \displaystyle 2\left[-\frac{\left(1-y\right)^2}{2}\right]\vert_{0}^{\frac{1}{2}}
                    \\
                    \\
                    \displaystyle \int_{0}^{\frac{1}{2}} 2\left(1-y\right) dy & = & \displaystyle 2\left[-\frac{\left(1-\frac{1}{2}\right)^2}{2}+ \frac{\left(1-0\right)^2}{2}\right]
                    \\
                    \\
                    \displaystyle \int_{0}^{\frac{1}{2}} 2\left(1-y\right) dy & = & \displaystyle 2\left[-\frac{\left(1-\frac{1}{2}\right)^2}{2}+ \frac{1}{2}\right]
                    \\
                    \\
                    \displaystyle \int_{0}^{\frac{1}{2}} 2\left(1-y\right) dy & = & \displaystyle 2\left[-\frac{\frac{1}{4}}{2}+ \frac{1}{2}\right]
                    \\
                    \\
                    \displaystyle \int_{0}^{\frac{1}{2}} 2\left(1-y\right) dy & = & \displaystyle -\frac{1}{4}+1
                    \\
                    \\
                    \displaystyle \int_{0}^{\frac{1}{2}} 2\left(1-y\right) dy & = &\frac{3}{4}
                \end{array}
                &
                \begin{array}{l}
                    \mbox{Calculos adicionales :}
                    \\
                    \begin{array}{rcl}
                        \displaystyle\int_{}^{} \left(1-y\right) dy & = & \displaystyle\int_{}^{}u (-1) du
                        \\
                        \\
                        & = &  \displaystyle -\int_{}^{}u du
                        \\
                        \\
                        & = & \displaystyle - \frac{u^2}{2}
                        \\
                        \\
                        & = & \displaystyle -\frac{\left(1-y\right)^2}{2}
                    \end{array}
                \end{array}
            \end{array}
        \end{equation*}
    \end{center}

    \begin{flushleft}
        Entonces podemos decir que $P \left(X > 0 , Y  < \frac{1}{2}\right) = \frac{3}{4}$
    \end{flushleft}

    \subsection*{Respuesta del (inciso 2-c)}
    \begin{flushleft}
        Dada la ecuaci\'on : \fbox{$V(X) = EX^2 - (EX)^2$} Para calcular EX nacesitamos calcular $ f_X (x) $ y tenemos que :
    \end{flushleft}
  
    \begin{equation*}
        \fbox{$
            \displaystyle f_X(x)=\int_{-\infty}^{+\infty} f\left(x,y\right) dy
        $} 
    \end{equation*}
          
    \begin{equation}
        \begin{array}{rcl}
            f_X(x) & = & \displaystyle  \int_{0}^{\left(1-x\right)} 2 dy
            \\
            \\
            f_X(x) & = & \displaystyle 2\int_{0}^{\left(1-x\right)}dy
            \\
            \\
            f_X(x) & = & \displaystyle 2\left(1-x\right)
        \end{array}
    \end{equation}
    
    \begin{flushleft}
        Vamos a calcular ahora $EX$ para esto necesitamos  :  
    \end{flushleft}
     
    \begin{equation*}
        \fbox{$
            EX = \displaystyle\int_{-\infty}^{+\infty}x f(x) dx
        $}
    \end{equation*}
      
    
    \begin{flushleft}
        Pero en el dominio de nuestra funci\'on tenemos : 
    \end{flushleft}
      
    \begin{center}
        \begin{equation*}
            \begin{array}{c|c}
                \begin{array}{rcl}
                    EX & = & \displaystyle\int_{0}^{1}2x \left(1-x\right) dx
                    \\
                    \\
                    EX & = &  2 \displaystyle\int_{0}^{1}x - x^2 dx
                    \\
                    \\
                    EX & = & \displaystyle 2\left[\frac{x^2}{2}- \frac{x^3}{3}\right]\vert_{0}^{1}
                    \\
                    \\
                    EX & = & \displaystyle 2\left[\frac{(1)^2}{2}- \frac{(1)^3}{3}\right]
                    \\
                    \\
                    EX & = & \displaystyle 2\left[\frac{1}{2}- \frac{1}{3}\right]
                    \\
                    \\
                    EX & = & \displaystyle 1-\frac{2}{3}
                    \\
                    \\
                    EX & = & \displaystyle \frac{1}{3}
                \end{array}
                &
                \begin{array}{l}
                    \mbox{Calculos adicionales :}
                    \\
                    \begin{array}{rcl}
                        \displaystyle \int_{}^{}\left(x-x^2\right)dx & = & \displaystyle\int_{}^{}x dx - \int_{}^{}x^2 dx
                        \\
                        \\
                        \displaystyle\int_{}^{}\left(x-x^2\right)dx & = & \displaystyle\frac{x^2}{2} - \frac{x^3}{3}
                        \vspace{1cm}
                        \\
                        \displaystyle\int_{}^{}\left(x^2x^3\right)dx & = & \displaystyle\int_{}^{}x^2dx-\int_{}^{}x^3dx
                        \\
                        \\
                        \displaystyle\int_{}^{}\left(x^2x^3\right)dx & = & 
                        \frac{x^3}{3}-\frac{x^4}{4}
                    \end{array}
                \end{array}
            \end{array}
        \end{equation*}
    \end{center}

    \begin{center}
        \begin{equation*}
            \begin{array}{rcl}
                EX^2 & = & \displaystyle\int_{-\infty}^{+\infty}x^2 f(x) dx
                \\
                \\
                EX^2 & = &  \displaystyle\int_{0}^{1}x^2 (2\left(1-x\right)) dx
                \\
                \\
                EX^2 & = & \displaystyle\int_{0}^{1}x^2 \left(1-x\right) dx
                \\
                \\
                EX^2 & = & \displaystyle 2\int_{0}^{1}x^2 -x^3 dx
                \\
                \\
                EX^2 & = & \displaystyle 2 \int_{0}^{1}\left[\frac{x^3}{3}- \frac{x^4}{4}\right]\vert_{0}^{1}
                \\
                \\
                EX^2 & = & \displaystyle 2\left[\frac{1}{3}-\frac{1}{4}\right] 
                \\
                \\
                EX^2 & = & \displaystyle \frac{2}{3} - \frac{1}{2}
                \\
                \\
                EX^2 & = & \displaystyle \frac{4-3}{6} 
                \\
                \\
                EX^2 & = & \displaystyle \frac{1}{6} 
            \end{array}
        \end{equation*}
    \end{center}

    \begin{flushleft}
        Ahora calculamos $V(X)$  , tenemos :
    \end{flushleft}
       
    \begin{equation*}
        \displaystyle V(X) = EX^2 - (EX)^2 = \frac{1}{6}-\left(\frac{1}{3}\right)^2= \frac{1}{6} - \frac{1}{9} = \frac{3-2}{18} = \frac{1}{18}
    \end{equation*}
        
    \begin{flushleft}
        Por lo tanto : $\displaystyle V(X) = \frac{1}{18}$
    \end{flushleft} 
        

    \section*{Ejercicio 3}
    
    \begin{flushleft}
        Un vector aleatorio con funci\'on de densidad conjunta.
    \end{flushleft}
   
    \begin{equation*} 
        f\left(x,y\right) = \begin{cases} 
        \mbox{$\displaystyle\frac{y}{2x} $ si} & 1\leq x\leq e, 0\leq y \leq a  
        \\
        0  & \mbox{en otro caso }
        \end{cases}
    \end{equation*}


    \subsection*{Respuesta del (inciso 3-a)}
    
    \begin{flushleft}
        Detremine el valor de la constante $a$ :
    \end{flushleft}
    
    \begin{flushleft}
        Tenemos que :
    \end{flushleft}
    
    \begin{equation*}
        \fbox{$
            \displaystyle\int_{-\infty}^{+\infty}\int_{-\infty}^{+\infty}f\left(u,v\right)dudv = 1
        $}
    \end{equation*}
    
    \begin{flushleft}
        Para el dominio de nuestra funcion tenemos : 
    \end{flushleft}
    
    \begin{equation*}
        \displaystyle\int_{1}^{e}\int_{0}^{a}\frac{y}{2x}dydx = 1
    \end{equation*}
   
    \begin{flushleft}
        Trabajando con la ecuaci\'on :
    \end{flushleft}
    
    \begin{equation*}
        \begin{array}{c|c}
           \begin{array}{rcl}
                \displaystyle\int_{1}^{e}\left(\int_{0}^{a} \frac{y}{2x}dy\right)dx & = & \displaystyle \int_{1}^{e}\frac{a^2}{4x}dx
                \\
                \\
                \displaystyle\int_{1}^{e}\left(\int_{0}^{a} \frac{y}{2x}dy\right)dx & = & \displaystyle\frac{a^2}{4}\int_{1}^{e}\frac{dx}{x}
                \\
                \\
                \displaystyle\int_{1}^{e}\left(\int_{0}^{a} \frac{y}{2x}dy\right)dx & = & \displaystyle\frac{a^2}{4}[\ln e]
                \\
                \\
                \displaystyle\int_{1}^{e}\left(\int_{0}^{a} \frac{y}{2x}dy\right)dx & = & \displaystyle\frac{a^2}{4}
            \end{array}
            
            &
            
            \begin{array}{c}
                \mbox{Calculos adicionales}
                \\
                \begin{array}{rcl}
                    \displaystyle\int_{0}^{a}\frac{y}{2x}dy & = & \displaystyle\frac{1}{2x}\int_{0}^{a}y dy
                    \\
                    \\
                    \displaystyle\int_{0}^{a}\frac{y}{2x}dy & = & \displaystyle\frac{1}{2x}\left[\frac{y^2}{2}\right]\vert_{0}^{a}
                    \\
                    \\
                    \displaystyle\int_{0}^{a}\frac{y}{2x}dy & = & \displaystyle\frac{1}{2x} \frac{a^2}{2}
                    \\
                    \\
                    \displaystyle\int_{0}^{a}\frac{y}{2x}dy & = & \displaystyle\frac{a^2}{4x}
                \end{array}
            \end{array}
        \end{array}
    \end{equation*}

    \begin{flushleft}
        Entonces tenemos :
    \end{flushleft}
    
    \begin{equation*}
        \begin{array}{rcl}
            \displaystyle \int_{1}^{e}\int_{0}^{a}\frac{y}{2x}dydx & = & 1
            \\
            \\
            \displaystyle \frac{a^2}{4} & = & 1
            \\
            \\
            \displaystyle a^2 & = & 4
            \\
            \\
            \displaystyle a & = &\pm2
        \end{array}
    \end{equation*}


    \begin{flushleft}
        Como tenemos que  $ a\geq0 $ porque $ 0\leq y \leq a $ por lo tanto : \fbox{$a = 2$}
    \end{flushleft}

    \subsection*{Respuesta del (inciso 3-b)}
    
    \begin{flushleft}
        Halla la funci\'on de densidad marginal de $X$ y $Y$.
    \end{flushleft}

    \begin{equation*}
        \begin{array}{c|c}
		    \begin{array}{rcl}
                f_{X}(x) & = &  \displaystyle\int_{-\infty}^{+\infty} f\left(x,y\right) dy
                \\
                \\
                f_{X}(x) & = &  \displaystyle\int_{0}^{2} \frac{y}{2x}dy
                \\
                \\
                f_{X}(x) & = &  \displaystyle\frac{1}{2x}\int_{0}^{2}ydy
                \\
                \\
                f_{X}(x) & = &  \displaystyle\frac{1}{2x}\left[\frac{y^2}{2}\right]\vert_{0}^{2}
                \\
                \\
                f_{X}(x) & = &\displaystyle\frac{1}{2x}* \frac{2^2}{2}
                \\
                \\
                f_{X}(x) & = &  \displaystyle \frac{1}{x}
		    \end{array}
		    &
	        \begin{array}{rcl}
                f_{Y}(y) & = &  \displaystyle\int_{1}^{e}\frac{y}{2x} dx
                \\
                \\
                f_{Y}(y) & = &  \displaystyle\frac{y}{2} \int_{1}^{e}\frac{1}{x}dx
                \\
                \\
                f_{Y}(y) & = &  \displaystyle\frac{y}{2} \left[\ln(x)\right]\vert_{1}^{e}
                \\
                \\
                f_{Y}(y) & = &   \displaystyle\frac{y}{2} \left[\ln(e) - \ln(1)\right]
                \\
                \\
                f_{Y}(y) & = &  \displaystyle\frac{y}{2} \left[1-0\right]
                \\
                \\
                f_{Y}(y) & = &  \displaystyle\frac{y}{2}
	       \end{array}
        \end{array}
    \end{equation*}

    \begin{flushleft}
        Entonces la funci\'on de densidad marginal de $X$ es     
    \end{flushleft}

    \begin{equation*}
        f_X\left(x\right) = \begin{cases} 
            \frac{1}{x} & 1\leq x \leq e  
            \\
            0  & \mbox{en otro caso }
            \end{cases}
    \end{equation*}

    \begin{flushleft}
        Y la funci\'on de densidad marginal de $Y$ es     
    \end{flushleft}

    \begin{equation*}
        f_Y\left(y\right) = \begin{cases} 
            \frac{y}{2} & 0 \leq y \leq 2  
            \\
            0  & \mbox{en otro caso }
            \end{cases}
    \end{equation*}

    \subsection*{Respuesta del (inciso 3-c)} 

    \begin{flushleft}
        Calcula $P(X >Y)$.
    \end{flushleft}
    
    \begin{flushleft}
        Partiendo de la f\'ormulas :
    \end{flushleft}

    \begin{equation*}
        \fbox{$
            \displaystyle\int_{1}^{2}\int_{0}^{x}\frac{y}{2x}dy dx  + \int_{2}^{e}\int_{0}^{2}\frac{y}{2x}dy dx 
        $}
    \end{equation*}
     
        
    \begin{center}
        \begin{equation*}
            \begin{array}{c|c}
                \begin{array}{rcl}
                    & \mbox{ Primero calculamos  :  \fbox{$\displaystyle\int_{1}^{2}\int_{0}^{x}\frac{y}{2x}dydx$}}&
                    \\
                    \\
                    & \displaystyle\int _{0}^{x}\frac{y}{2x}dy  = \displaystyle\frac{1}{2x}\int_{0}^{x}ydy &
                    \\
                    \\
                    & \displaystyle\int_{0}^{x}\frac{y}{2x}dy  =  \frac{1}{2x}\left[\frac{y^2}{2}\right]\vert_{0}^{x} &
                    \\
                    \\
                    &\displaystyle \int_{0}^{x}\frac{y}{2x}dy = \frac{1}{2x}\left[\frac{x^2}{2}\right] &
                    \\
                    \\
                    &\displaystyle\int_{0}^{x}\frac{y}{2x}dy  =  \frac{x}{4} &
                    \\
                    \\
                    &\displaystyle\mbox{Y ahora sustisuimos :} &
                    \\
                    \\
                    & \displaystyle \int_{1}^{2}\int_{0}^{x}\frac{y}{2x}dy dx  =  \int_{1}^{2} \frac{x}{4}dx &
                    \\
                    \\
                    &\displaystyle\int_{1}^{2}\int_{0}^{x}\frac{y}{2x}dy dx  =  \frac{1}{4}\int_{1}^{2}x dx  &
                    \\
                    \\
                    &\displaystyle\int_{1}^{2}\int_{0}^{x}\frac{y}{2x}dy dx  =  \frac{1}{4}\left[\frac{x^2}{2}\right]\vert_{0}^{x} &
                    \\
                    \\
                    & \displaystyle \int_{1}^{2}\int_{0}^{x}\frac{y}{2x}dy dx  =           \frac{1}{4}\left[\frac{4}{2}-\frac{1}{2}\right] &
                    \\
                    \\
                    &\displaystyle \int_{1}^{2}\int_{0}^{x}\frac{y}{2x}dy dx  =  \frac{1}{4}\left[\frac{3}{2}\right] &
                    \\
                    \\
                    & \displaystyle \int_{1}^{2}\int_{0}^{x}\frac{y}{2x}dy dx  =  \frac{3}{8}&
                    
                \end{array}
                &
                \begin{array}{rcl}
                    &\mbox{Segundo calculamos :\fbox{$\displaystyle\int_{2}^{e}\int_{0}^{2}\frac{y}{2x}dydx $}}&
                    \\
                    \\
                    & \displaystyle\int_{0}^{2}\frac{y}{2x}dy = \frac{1}{2x}\int_{0}^{2}ydy&
                    \\
                    \\
                    & \displaystyle\int_{0}^{2}\frac{y}{2x}dy = \frac{1}{2x}\left[\frac{y^2}{2}\right]\vert_{0}^{2}&
                    \\
                    \\
                    & \displaystyle\int_{0}^{2}\frac{y}{2x}dy = \frac{1}{2x} * \frac{4}{2}&
                    \\
                    \\
                    & \displaystyle\int_{0}^{2}\frac{y}{2x}dy = \frac{1}{x} &
                    \\
                    \\
                    &\displaystyle\displaystyle\mbox{Y ahora sustisuimos :} &
                    \\
                    \\
                    &\displaystyle\int_{2}^{e}\int_{0}^{2}\frac{y}{2x}dydx = \int_{2}^{e}\frac{1}{x}dx &
                    \\
                    \\
                    &\displaystyle\int_{2}^{e}\int_{0}^{2}\frac{y}{2x}dydx = \ln x\vert_{2}^{e} &
                    \\
                    \\
                    &\displaystyle\int_{2}^{e}\int_{0}^{2}\frac{y}{2x}dydx = \ln(e)-\ln(2) &
                    \\
                    \\
                    &\displaystyle\int_{2}^{e}\int_{0}^{2}\frac{y}{2x}dydx = 1 -\ln(2) &
                \end{array}
            \end{array}
        \end{equation*}
    \end{center}
    \begin{flushleft}
        Y por \'ultimo sustituimos en la  ecuacui\'on original :
    \end{flushleft}

    \begin{equation*}
        \fbox{$\to\displaystyle\int_{1}^{2}\int_{0}^{x}\frac{y}{2x}dy dx  + \int_{2}^{e}\int_{0}^{2}\frac{y}{2x}dy dx = \frac{3}{8} + 1 - \ln(2) = \frac{11}{8}-\ln(2) $}
    \end{equation*}  
        
    \subsection*{Respuesta del (inciso 3-d)} 
    \begin{flushleft}
        Determinar $EX$  y $V(Y)$.
    \end{flushleft}
    \begin{center}
        \begin{equation*}
		    \begin{array}{c|c}
		        \begin{array}{rcl}
                    EX & = & \displaystyle\int_{-\infty}^{+\infty} x f(x)dx
                    \\
                    \\
                    EX & = & \displaystyle\int_{1}^{e} x \frac{1}{x}dx
                    \\
                    \\
                    EX & = & \displaystyle\int_{1}^{e} dx
                    \\
                    \\
                    EX & = & \displaystyle x\vert{1}^{e}
                    \\
                    \\
                    EX & = & \displaystyle e-1
                \end{array}
                 &
                \begin{array}{rcl}
                    EY & = & \displaystyle\int_{-\infty}^{+\infty}yf(y)dy 
                    \\
                    \\
                    EY & = & \displaystyle\int_{0}^{2}y\frac{y}{2}dy 
                    \\
                    \\
                    EY & = & \displaystyle\int_{0}^{2}\frac{y^2}{2}dy 
                    \\
                    \\
                    EY & = & \displaystyle\frac{1}{2}\int_{0}^{2} y^2dy
                    \\
                    \\
                    EY & = & \displaystyle\frac{1}{2}\left[\frac{y^3}{3}\right]\vert_{0}^{2}
                    \\
                    \\
                    EY & = & \displaystyle\frac{1}{2} * \frac{2^3}{3}
                    \\
                    \\
                    EY & = & \displaystyle\frac{8}{6} 
                    \\
                    \\
                    EY & = & \displaystyle\frac{4}{3}
	            \end{array}
            \end{array}
        \end{equation*}
    \end{center}
         
    \begin{equation*}
		\begin{array}{rcl}
		    EY^2 & = & \displaystyle\int_{0}^{2}y^2\frac{y}{2}dy
		    \\
		    \\
		    EY^2 & = & \displaystyle\int_{0}^{2}\frac{y^3}{2}dy
		    \\
		    \\
		    EY^2 & = & \displaystyle\frac{1}{2}dy\int_{0}^{2}y^3 dy
		    \\
		    \\
		    EY^2 & = & \displaystyle\frac{1}{2}\left[\frac{y^4}{4}\right]\vert_{0}^{2}
		    \\
		    \\
		    EY^2 & = & \displaystyle\frac{1}{2}\left[\frac{(2)^4}{4}\right]
		    \\
		    \\
		    EY^2 & = & \displaystyle\frac{1}{2}\frac{16}{4}
		    \\
		    \\
		    EY^2 & = & \displaystyle\frac{1}{2}* 4
		    \\
		    \\
		    EY^2 & = & 2
		\end{array}
    \end{equation*}

    \begin{flushleft}
        Por lo tanto podemos calcular $V(Y)$:
    \end{flushleft}
        
    \begin{equation*}
        \begin{array}{rcl}
            V(Y) & = & \displaystyle EY^2 - (EY)^2
            \\
            \\
            V(Y) & = & \displaystyle 2 - \left(\frac{4}{3}\right)^2
            \\
            \\
            V(Y) & = & \displaystyle 2 - \frac{16}{9}
            \\
            \\
            V(Y) & = & \displaystyle\frac{18}{9} - \frac{16}{9}
            \\
            \\
            V(Y) & = & \displaystyle\frac{2}{9}
        \end{array}
    \end{equation*}

    \section*{Ejercicio 4 } 

    \begin{flushleft}
        Sea $\left(X,Y\right)$  un vector aleatorio con funcion de densidad conjunta 
    \end{flushleft}

    \begin{equation*}
        f\left(x,y\right) = \begin{cases}
            \frac{1}{2x^2} & \mbox{si $ | x | < 1 , 0 \leq y \leq x^2$}
            \\
            0 & \mbox{en otro caso}
        \end{cases}
    \end{equation*}

    \subsection*{a-) Halle las funciones de densidad marginal de $X$ y de $Y$}

    \begin{flushleft}
        Primero vamos a calcular $f_{X}\left(x\right)$ 
    \end{flushleft}

    \begin{equation*}
        \begin{array}{rcl}
            \displaystyle f_{X}\left(x\right) & = & \displaystyle \int_{-\infty}^{+ \infty} f \left(x,y\right) \,\mathrm{d}y
            \\
            \\
            & =  & \displaystyle \int_{0}^{x^2} \frac{1}{2x^2} \,\mathrm{d}y   
            \\
            \\
            & = & \displaystyle \frac{1}{2x^2} \left[y\right] \vert_{0}^{x^2}
            \\
            \\
            & = & \displaystyle \frac{x^2}{2x^2}
            \\
            \\
            \displaystyle f_{X}\left(x\right) & = & \frac{1}{2} 
        \end{array}
    \end{equation*}

    \begin{flushleft}
        Ahora vamos a calcular $f_{Y} \left(y\right)$
    \end{flushleft}

    

    \begin{flushleft}
        Si hacemos lo mismo que  con $Y$ obtenemos que : 
    \end{flushleft}
         
    \begin{equation*}
        \begin{array}{rcl}
            f_Y\left(y\right) & = & \displaystyle \int_{- \infty}^{+ \infty} f \left(x,y \right) \,\mathrm{d}x 
            \\
            \\
            & = & \displaystyle  \frac{1}{\sqrt{y}} -1 
        \end{array}
    \end{equation*}


    \subsection*{b) Calcule  $P\left(Y > 0,25 \right) $y $P \left(X > −0.5, Y < 0.64\right)$.}
    
    \vspace{0.5cm}
    \begin{flushleft}
        Vamos primero a calcular $P \left(Y > \frac{1}{4}\right)$ , tenemos : 
    \end{flushleft}

    \begin{equation*}
        P\left(Y>\frac{1}{4}\right) = \displaystyle \int_{-1}^{- \frac{1}{2}} \int_{\frac{1}{4}}^{x^2} \frac{1}{2x^2} \,\mathrm{d}y  \,\mathrm{d}x  + \int_{1}^{2} \int_{\frac{1}{4}}^{x^2} \frac{1}{2x^2} \,\mathrm{d}y  \,\mathrm{d}x 
    \end{equation*}

    \begin{flushleft}
        Vamos a calcular $\displaystyle \int_{-1}^{- \frac{1}{2}} \int_{\frac{1}{4}}^{x^2} \frac{1}{2x^2} \,\mathrm{d}y  \,\mathrm{d}x $
    \end{flushleft}

    \begin{equation*}
        \begin{array}{rcl}
            \displaystyle \int_{-1}^{- \frac{1}{2}} \int_{\frac{1}{4}}^{x^2} \frac{1}{2x^2} \,\mathrm{d}y  \,\mathrm{d}x  & = & \displaystyle  \frac{1}{2} \int_{-1}^{- \frac{1}{2}} \left(\int_{\frac{1}{4}}^{x^2}x^{-2}\,\mathrm{d}y \right) \,\mathrm{d}x 
            \\
            \\
            && \mbox{tenemos que: $\displaystyle \int_{\frac{1}{4}}^{x^2}x^{-2}\,\mathrm{d}y  = 1 - \frac{1}{4} x^{-2}$} 
            \\
            && \mbox{por lo tanto : }
            \\
            \\
            & = & \displaystyle  \frac{1}{2} \int_{-1}^{- \frac{1}{2}} \left(1 - \frac{1}{4} x^{-2}\right) \,\mathrm{d}x 
            \\
            \\
            && \mbox{tenemos : $\displaystyle  \int_{-1}^{- \frac{1}{2}} \left(1 - \frac{1}{4} x^{-2}\right) \,\mathrm{d}x = \frac{1}{4} $}
            \\
            \mbox{Por lo tanto: }
            \\
            \\
            \displaystyle \int_{-1}^{- \frac{1}{2}} \int_{\frac{1}{4}}^{x^2} \frac{1}{2x^2} \,\mathrm{d}y  \,\mathrm{d}x & = & \displaystyle \left(\frac{1}{2} \right) \left(\frac{1}{4}\right) = \frac{1}{8}
        \end{array}
    \end{equation*}

    \begin{flushleft}
        vamos ahora a calcular : $\displaystyle \int_{1}^{2} \int_{\frac{1}{4}}^{x^2} \frac{1}{2x^2} \,\mathrm{d}y  \,\mathrm{d}x$
    \end{flushleft}


    \begin{equation*}
        \begin{array}{rcl}
            \displaystyle \int_{\frac{1}{2}}^{1} \int_{\frac{1}{4}}^{x^2} \frac{1}{2x^2} \,\mathrm{d}y  \,\mathrm{d}x  & = & \displaystyle  \frac{1}{2} \int_{\frac{1}{2}}^{1} \left(\int_{\frac{1}{4}}^{x^2}x^{-2}\,\mathrm{d}y \right) \,\mathrm{d}x 
            \\
            \\
            && \mbox{tenemos que: $\displaystyle \int_{\frac{1}{4}}^{x^2}x^{-2}\,\mathrm{d}y  = 1 - \frac{1}{4} x^{-2}$} 
            \\
            && \mbox{por lo tanto : }
            \\
            \\
            & = & \displaystyle  \frac{1}{2} \int_{\frac{1}{2}}^{1} \left(1 - \frac{1}{4} x^{-2}\right) \,\mathrm{d}x 
            \\
            \\
            && \mbox{tenemos : $\displaystyle  \int_{\frac{1}{2}}^{1} \left(1 - \frac{1}{4} x^{-2}\right) \,\mathrm{d}x = \frac{1}{4} $}
            \\
            \mbox{Por lo tanto: }
            \\
            \\
            \displaystyle \int_{\frac{1}{2}}^{1} \int_{\frac{1}{4}}^{x^2} \frac{1}{2x^2} \,\mathrm{d}y  \,\mathrm{d}x & = & \displaystyle \left(\frac{1}{2} \right) \left(\frac{1}{4}\right) = \frac{1}{8}
        \end{array}
    \end{equation*}

    \begin{flushleft}
        Podemos concluir entonces que : 
    \end{flushleft}

    \begin{equation*}
        \begin{array}{rcl}
            P\left(Y>\frac{1}{4}\right) &= &  \displaystyle \int_{-1}^{- \frac{1}{2}} \int_{\frac{1}{4}}^{x^2} \frac{1}{2x^2} \,\mathrm{d}y  \,\mathrm{d}x  + \int_{1}^{2} \int_{\frac{1}{4}}^{x^2} \frac{1}{2x^2} \,\mathrm{d}y  \,\mathrm{d}x         
            \\
            \\
            P\left(Y>\frac{1}{4}\right) &= & \displaystyle  \frac{1}{8} + \frac{1}{8} = \frac{1}{4}
        \end{array}
    \end{equation*}

    \vspace{1cm}

    \begin{flushleft}
        Vamos ahora a calcular $P\left(X> - 0.5 , Y < 0.64\right)$ , tenemos :    
    \end{flushleft}

    \begin{equation*}
        P\left(X> - \frac{1}{2} , Y < 0.64\right) = \displaystyle \int_{-\frac{1}{2}}^{\frac{4}{5}} \int_{0}^{x^2} \frac{1}{2x^2} \,\mathrm{d}y  \,\mathrm{d}x  + \displaystyle \int_{\frac{4}{5}}^{1} \int_{0}^{0.64} \frac{1}{2x^2} \,\mathrm{d}y \,\mathrm{d}x
    \end{equation*}

    \begin{equation*}
        \begin{array}{rcl}
            \displaystyle \int_{-\frac{1}{2}}^{\frac{4}{5}} \int_{0}^{x^2} \frac{1}{2x^2} \,\mathrm{d}y  \,\mathrm{d}x  & = & \displaystyle \frac{1}{2} \int_{-\frac{1}{2}}^{\frac{4}{5}} \left(\int_{0}^{x^2} x^{-2} \,\mathrm{d}y \right)  \,\mathrm{d}x 
            \\
            \\
            && \mbox{tenemos que  $\displaystyle \int_{0}^{x^2} x^{-2} \,\mathrm{d}y  = 1 $}
            \\
            && \mbox{por lo tanto } 
            \\
            \\
            \displaystyle \int_{-\frac{1}{2}}^{\frac{4}{5}} \int_{0}^{x^2} \frac{1}{2x^2} \,\mathrm{d}y  \,\mathrm{d}x  & = & \displaystyle \frac{1}{2} \int_{-\frac{1}{2}}^{\frac{4}{5}}   \,\mathrm{d}x  
            \\
            \\
            & = & \displaystyle \frac{1}{2} \left(\left[x\right] \vert_{-\frac{1}{2}}^{\frac{4}{5}} \right)
            \\
            \\
            & = & \displaystyle \frac{1}{2} \left(\frac{4}{5} + \frac{1}{2}\right) = \left(\frac{1}{2}\right) \left(\frac{13}{10}\right) 
            \\
            \\
            \displaystyle \int_{-\frac{1}{2}}^{\frac{4}{5}} \int_{0}^{x^2} \frac{1}{2x^2} \,\mathrm{d}y  \,\mathrm{d}x & = & \displaystyle \frac{13}{20} 
        \end{array}
    \end{equation*}

    \hspace{1cm}

    \begin{flushleft}
        ahora calculamos la $ \displaystyle \int_{\frac{4}{5}}^{1} \int_{0}^{0.64} \frac{1}{2x^2} \,\mathrm{d}y \,\mathrm{d}x $
    \end{flushleft}

    \begin{equation*}
        \begin{array}{rcl}
            \displaystyle \int_{\frac{4}{5}}^{1} \int_{0}^{0.64} \frac{1}{2x^2} \,\mathrm{d}y \,\mathrm{d}x & = & \displaystyle  \frac{1}{2} \int_{\frac{4}{5}}^{1} \left(\int_{0}^{0.64} x^{-2} \,\mathrm{d}y \right) \,\mathrm{d}x 
            \\
            \\
            && \mbox{tenemos que $ \displaystyle \int_{0}^{0.64} x^{-2} \,\mathrm{d}y = \frac{64}{100}x^{-2} $}
            \\
            && \mbox{por lo tanto} 
            \\
            \\
            \displaystyle \int_{\frac{4}{5}}^{1} \int_{0}^{0.64} \frac{1}{2x^2} \,\mathrm{d}y \,\mathrm{d}x & = & \displaystyle  \frac{1}{2}\displaystyle \int_{\frac{4}{5}}^{1} \left(\frac{64}{100} x^{-2}\right) \,\mathrm{d}x
            \\
            \\
            & = & \displaystyle \left(\frac{1}{2}\right) \left(\frac{64}{100}\right) \int_{\frac{4}{5}}^{1} x^{-2} \,\mathrm{d}x 
            \\
            \\
            && \mbox{tenemos que $ \displaystyle \int_{\frac{4}{5}}^{1} x^{-2} \,\mathrm{d}x$} = \frac{1}{4}
            \\
            && \mbox{por lo tanto } 
            \\
            \\
            & = & \displaystyle \left(\frac{1}{2}\right) \left(\frac{64}{100}\right) \left(\frac{1}{4}\right) = \left(\frac{1}{2}\right) \left(\frac{16}{100}\right)
            \\
            \\
            \displaystyle \int_{\frac{4}{5}}^{1} \int_{0}^{0.64} \frac{1}{2x^2} \,\mathrm{d}y \,\mathrm{d}x & = & \displaystyle \frac{8}{100} = 0.08
        \end{array}
    \end{equation*}

    \begin{flushleft}
        Por lo que podemos concluir que : 
    \end{flushleft}

    \begin{equation*}
        P\left(X> - \frac{1}{2} , Y < 0.64\right) = \frac{13}{20} + \frac{8}{100} = \frac{73}{100} = 0.73
    \end{equation*}


    \subsection*{c) Diga si las variables son independientes.}

    \begin{flushleft}
        Las variables no son independientes pues no se cumple : 
    \end{flushleft}

    \begin{equation*}
        f_{\left(X,Y\right)}\left(x,y\right) = f_X\left(x\right) f_Y\left(y\right) 
    \end{equation*}

    \vspace{2cm} 
    \section*{Ejercicio 5 }

    \subsection*{Determinar el valor de la constante $c$}

    \begin{flushleft}
        Sabemos que $ \displaystyle \int_{-\infty}^{+\infty} \int_{-\infty}^{+\infty} f\left(x,y\right) \,\mathrm{d}x  \,\mathrm{d}y   =1 $
    \end{flushleft}

    \begin{flushleft}
        Para nuestro caso tenemos que $\displaystyle \int_{0}^{2} \int_{0}^{x} c\left(x+y \right) \,\mathrm{d}y \,\mathrm{d}x   =1 $ , entonces : 
    \end{flushleft}

    \begin{equation*}
        \begin{array}{rcl}
            \displaystyle \int_{0}^{2} \int_{0}^{x} c\left(x+y \right) \,\mathrm{d}y \,\mathrm{d}x  & = & \displaystyle c  \int_{0}^{2} \left( \int_{0}^{x} \left(x+y \right) \,\mathrm{d}y \right)\,\mathrm{d}x 
            \\
            \\
            && \mbox{pero la $ \displaystyle  \int_{0}^{x} \left(x+y \right) \,\mathrm{d}y = x^2 + \frac{x^2}{2} $}
            \\
            && \mbox{por lo que tenemos : }
            \\
            \\
            & = &  \displaystyle c  \int_{0}^{2} \left(x^2 + \frac{x^2}{2}\right)\,\mathrm{d}x 
            \\
            \\
            && \mbox{tenemos que $ \displaystyle \int_{0}^{2} \left(x^2 + \frac{x^2}{2}\right)\,\mathrm{d}x  = \frac{10}{3}$ }
            \\
            \mbox{Por lo que tenemos  :}&&
            \\
            \\
            \displaystyle \int_{0}^{2} \int_{0}^{x} c\left(x+y \right) \,\mathrm{d}y \,\mathrm{d}x  & = & \displaystyle c \left(\frac{10}{3}\right)
            \\
            \\
            \mbox{Si sustituimos aqui }&&
            \\
            \\
            1 & = & \displaystyle c \left(\frac{10}{3}\right)
        \end{array}
    \end{equation*}

    \begin{flushleft}
        Por lo que nos queda que : $ \displaystyle c = \frac{3}{10}$
    \end{flushleft}




    \section*{Ejercicio 7} 

    \begin{flushleft}
        Como $X$,$Y$ y $Z$ son i.i.d  con distribuci\'on uniforme sobre el intervalo  $\left(0,1\right)$, entonces :
    \end{flushleft}
    
    \begin{equation*}
    f_{X,Y,Z}\left(x,y,z\right) = f_X\left(x\right) f_Y\left(y\right) f_Z\left(z\right) \hspace{1cm}0\leq x \leq 1\leq y \leq 1\leq z \leq 1
    \end{equation*}
    
    \begin{equation*}
        \begin{array}{rcl}
            \displaystyle P\left(Z>XY\right) & = & \displaystyle \iiint_{z>xy} f_{X,Y,Z} dzdydx
            \\
            \\
            & = & \displaystyle\int_{0}^{1}\int_{0}^{1}\int_{xy}^{1}  dzdydx
            \\
            \\
            & = & \displaystyle\int_{0}^{1}\int_{0}^{1}\left(1-xy\right)dydx
            \\
            \\
            & = &  \displaystyle\int_{0}^{1}\frac{-x+2}{2}dx
            \\
            \\
            & = & \displaybreak \frac{3}{4}
        \end{array}
    \end{equation*}

    \begin{flushleft}
        La probabilidad $P\left(Z > XY\right)$ es de $\frac{3}{4}$
    \end{flushleft}


    \section*{Ejercicio 8} Tenemos que $ X y Y$ son variables aleatorias continuas independientes con funciones de densidad $f_{X}(x) y f_{Y}(y)$
    
    \begin{flushleft}
        Primero la funcion de distribuci\'on de $X + Y$ la obtenemos por :
    \end{flushleft}

    \begin{equation*}
        \begin{array}{rcl}
            F_{X + Y}(a) & = & \displaystyle P\{X+Y \leq a\}
            \\
            \\
            F_{X + Y}(a) & = & \displaystyle\iint_{X+Y \leq a} f_X(x)f_Y(y)dxdy
            \\
            \\
            F_{X + Y}(a) & = & \displaystyle\int_{-\infty}^{+\infty}\int_{-\infty}^{a-y} f_X(x)f_Y(y)dxdy
            \\
            \\
            F_{X + Y}(a) & = & \displaystyle\int_{-\infty}^{+\infty}\int_{-\infty}^{a-y} f_X(x)dx f_Y(y)dy
            \\
            \\
            F_{X + Y}(a) & = & \displaystyle\int_{-\infty}^{+\infty} F_X(a-y) f_Y(y)dy
        \end{array}
    \end{equation*}

    \begin{flushleft}
        Ahora como tenemos la funci\'on de distribuci\'on de $X+Y$ podemos entoces encontrar la funci\'on de densidad $f_{X+Y}$ de $ X+Y$ de la siguiente forma :
    \end{flushleft}
     
    \begin{equation*}
        \begin{array}{rcl}
            f_{X + Y}(a) & = & \displaystyle\frac{d}{da}\int_{-\infty}^{+\infty}F_X\left(a-y\right)f_Y(y)dy
            \\
            \\
            f_{X + Y}(a) & = & \displaystyle\int_{-\infty}^{+\infty}\frac{d}{da}F_X\left(a-y\right)f_Y(y)dy
            \\
            \\
            f_{X + Y}(a) & = & \int_{-\infty}^{+\infty}f_X\left(a-y\right)f_Y(y)dy
        \end{array}
     \end{equation*}









\end{document}